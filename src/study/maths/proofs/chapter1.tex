\chapter{Intuitive Proofs}


\begin{principle}[The Pigeonhole Principle]
  \label{pigeonhole}
If $kn+1$ objects are placed into $n$ boxes,
then at least one box has at least $k+1$ objects.
\end{principle}

\begin{proposition}
  Given any $101$ integers frrom ${1, 2, 3, ..., 200}$, at least one of these numbers will divide another.
\end{proposition}

\begin{scratch}
Since there are $101$ items, we can consider the pigeon hole principle with $k=1$ and $n=100$.

Let us consider the following boxes. Create a box for each of the odd numbers ${1, 3, 5, ..., 199}$ and for any number $x$ if $x$ is of the form $x = 2^k \cdot m$, where $m$ is odd and $k >= 0$, we can put $x$ in the box $m$.

There are $100$ odd numbers in the set so we have $100$ boxes. And any two numbers in a box only differ by $2^k$ for some $k$. Thus, for any two numbers in one box, the smaller number divides the larger one.

For any odd number larger than $101$, it will be the only number in that box.
\end{scratch}

\begin{proof}
  For each number $n$ from the set ${1, 2, 3, ..., 200}$, write it in the form of $n = 2^k \cdot m$ where $k >= 0$ and $m$ is an odd number.

  Now, create a box for each odd number from $1$ to $199$. There will be $100$ such boxes. For each of the given $101$ integers, \\

  If $n = 2^k \cdot m$ then put $n$ in the box numbered $m$. \\

  Since $101$ integers are placed in $100$ boxes, there must be at least one box with more than $1$ integer by $\ref{pigeonhole}$.

  Suppose the box $m$ contains two numbers of the form $n_1 = 2^k \cdot m$ and $n_2 = 2^l \cdot m$ where without loss of generality $k > l$. Then we can show that
  $$\frac{n_1}{n_2} = \frac{2^k \cdot m}{2^l \cdot m} = 2^{k-l}$$

  Here, $2^{k-l}$ is an integer since $k > l$, thus, $n_2$ divides $n_1$.

  Thus, proved.
\end{proof}

\begin{proposition}
  Suppose $G$ is a graph with $n \geq 2$ vertices. Then, $G$ contains two vertices which have the same degree.
\end{proposition}

\begin{proof idea}
  The possible degrees of a vertex is any number between $0$ and $n-1$.
  Thus, there are $n$ boxes for each possible value for the degree of a vertex and $n$ vertices.

  We can show that at least one box must be empty. Therefore, we need to put $n$ vertices in $n-1$ boxes and by The Pigeonhole Principe (\ref{pigeonhole}), there must be at least two vertices in the same box, i.e., have the same degree.

  We can show that both box $0$ and box $n-1$ cannot have a vertex because if vertex $v_1$ is in box $n-1$ then it has an edge connecting it to every other vertex.

  Thus, every other vertex has an edge connecting it to $v_1$ which implies that every other vertex has at least a degree of $1$ and box $0$ must be empty.

  If there is no vertex in box $n-1$ then we have box $n-1$ that is empty.

  Thus, at least one box is empty in both scenarios.
\end{proof idea}

\begin{proof}
  Let $G$ be a graph with $n \geq 2$ vertices. Create boxes numbered from $0$ to $n-1$.

  Now, for each vertex, let us say it's degree is $d$, then put that vertex in box $d$. Let us take box $0$ and $n-1$. Both of these boxes are either empty or have some vertex in them.

  \bigbreak
  \underline{Case 1}. Box $n-1$ is empty. \\
  Since box $n-1$ is empty, we have $n$ vertices being placed into $n-1$ boxes.
  Therefore, by The Pigeonhole Principle (\ref{pigeonhole}), there are at least one box with at least two vertices.

  Thus, there are at least two vertices with the same degrees.

  \bigbreak
  \underline{Case 2}. Box $n-1$ is not empty. \\
  The vertex in box $n-1$ must have a degree of $n-1$ which implies it has an edge connecting to $n-1$ vertices. \\
  Therefore, all $n$ vertices have at least one edge connecting them to another edge and all $n$ vertices have a degree of at least $1$.

  This implies that box $0$ must be empty since all vertices have a degree of at least $1$.

Since box $0$ is empty, there are $n$ vertices placed into $n-1$ boxes.

Therefore, by The Pigeonhole Principle (\ref{pigeonhole}), there are at least two vertices in the same box and have the same degree.
\bigbreak
Thus, proved.

\end{proof}


\begin{proposition}
  If you draw five points on the surface of an orange in marker, then there is always a way to cut the orange in half so that four points ( or some part of each of those points ) all lie on one of the halves.
\end{proposition}

\begin{scratch}
There are two subtle statements in the propsition. First it asserts that "always a way to cut the orange in half so that...".
If doesn't assert that \emph{any} such cut has this property.

Second, it is important that we say "or some part of each of those points". When you use a marker to make the points, the points are big enough that when you slice through any point, part of the point appears on \emph{both} halves.
\end{scratch}

\begin{named}[Classical Geometry Theorem]
  Given any two points on the sphere, there is a great circle that passes through those two points.
\end{named}


\begin{proof}
  Take $2$ out of $5$ given points. By Classical Geometry Theorem, there is a great circle passing through these points. Thus, this great circle divides that sphere in two halves.

  The remaining three points are placed among these two halves. Thus, by The Pigeonhole Principle (\ref{pigeonhole}), there are at least two points on one of the havles.

  Adding the two initially chosen points to both halves, we have one half with atleast four points.

  Hence, proved.
\end{proof}
