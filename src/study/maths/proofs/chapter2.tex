\chapter{Direct Proofs}

\begin{definition}
	\label{div}
	A nonzero integer $a$ is said to \emph{divide} an integer $b$ if $b = ak$ for some integer $k$. When $a$ divides $b$, we write $a \mid b$ and when $a$ does not divide $b$, we write $a \nmid b$.
\end{definition}

\begin{proposition} \label{transdiv}
	Let $a$, $b$ and $c$ be intgers. If $a \mid b$ and $b \mid c$, then $a \mid c$.
\end{proposition}

\begin{theorem}[The Division Algorithm] \label{divalgo}
	For all integers $a$ and $m$ witht $m > 0$, there exist unique integers $q$ and $r$ such that
	$$a = mq+r$$
	where $0 \leq r < m$.
\end{theorem}

\begin{definition} \label{gcd}
	Let $a$ and $b$ be integers. If $c \mid a$ and $c \mid b$, then $c$ is said to be a \emph{common divisor} of $a$ and $b$.

	The \emph{greatest common divisor} of $a$ and $b$ is the largest $d$ such that $d \mid a$ and $d \mid b$. This number is denoted $\gcd(a,b)$.
\end{definition}

\begin{theorem}[Bezout's Identity] \label{gcdalgo}
	If $a$ and $b$ are positive integers, then there exist integers $k$ and $l$ such that $$\gcd(a,b) = ak + bl$$
\end{theorem}

\begin{scratch}
	Let's jot down an example. Let $a = 12$ and $b = 20$, making $\gcd(a, b) = 4$.
	Then indeed we get, $$4 = 12*2 + 20*(-1)$$
	Or maybe $$4 = 12*(-3) + 20*2$$
\end{scratch}

\begin{proof}
	Assume $a$ and $b$ are fixed positive integers. Then the expression $ax+by$ can take infinitely many integer values for any integers $x$ and $y$. It can even be $0$ for $x = y = 0$.

	Let $d$ be the smallest positive integers that the expression $ax+by$ can take. And let $k$ and $l$ be integers for which
	\begin{equation} \label{eq:1}
		d = ak + bl
	\end{equation}

	Now, we need to prove that $d = \gcd(a,b)$. We will do this in two parts. First, we will show that $d$ is a common divisor of $a$ and $b$. Then, we will show that $d = \gcd(a,b)$.

	\underline{Part 1: $d$ is a common divisor of $a$ and $b$.}

	Since $d > 0$, therefore, by Theorem \ref{divalgo}, there exists integers $q$ and $r$ such that
	$$a = dq + r$$
	with $0 \leq r < d$. By rewriting this, we get,
	\begin{align}
		r & = a - dq           \\
		  & = a - (ak+bl)q     \\
		  & = a - akq - blq    \\
		  & = a(1-kq) + b(-lq)
	\end{align}

	Since $1-kq$ and $-lq$ are integers, we have found another expression of the form $ax + by$. But since $0 \leq r < d$ and $d$ was the smallest posititve integer of the form $ax+by$ then it must be true that $r = 0$.

	Therefore, we have the equation $a = dq + r$ which simplifies to $a = dq$. And thus, by Definition \ref{div}, $d \mid a$.

	Similarly, we can also prove that $d \mid b$.
	Thus, $d$ is a common divisor of $a$ and $b$.

	\bigbreak

	\underline{Part 2: $d$ is the greatest common divisor of $a$ and $b$.}

	Suppose that $d'$ is another common divisor of $a$ and $b$. Here, we must show that $d' \leq d$ for all such $d'$.

	By Definition \ref{div}, we have $$a = d'm \text{ and } b = d'n$$
	for some integers $m$ and $n$. Then, applying the above to Equation \ref{eq:1},
	\begin{align}
		d           & = ak + bl         \\
		            & = d'mk + d'nl     \\
		            & = d'(mk+nl)       \\
		\implies d' & = \frac{d}{mk+nl}
	\end{align}

	Since $mk + nl$ is an integer and $d$ is positive, this implies that $d' \leq d$.

	Note: If $mk+nl$ is negative then $d'$ is negative and it is trivial that $d' \leq d$. If $mk+nl$ is positive then $d' = \frac{d}{mk+nl}$ implies that $d' \leq d$.

	Thus, $d$ is in fact the greatest common divisor of $a$ and $b$, i.e., $$\gcd(a, b) = d = ak + bl$$

	Hence, proved.
\end{proof}

\begin{definition} \label{mod}
	For integers $a$, $r$ and $m$, we say that $a$ \emph{is congruent to $r$ modulo $m$}, and we write $a \equiv r \pmod m$, if $m \mid (a-r)$.
\end{definition}

\begin{proposition}[Properties of Modular Arithmetic] \label{modprop}
	Assume that $a$, $b$, $c$, $d$ and $m$ are integers, $a \equiv b \pmod m$ and $c \equiv d \pmod m$. Then,

	\begin{enumerate}
		\item $a + c \equiv b + d \pmod m$
		\item $a - c \equiv b - d \pmod m$
		\item $a \cdot c \equiv b \cdot d \pmod m$
	\end{enumerate}
\end{proposition}

\begin{definition} \label{prime}
	An integer $p \geq 2$ is \emph{prime} if its only positive divisors are $1$ and $p$. An integer $p \geq 2$ is \emph{composite} if it is not prime.

	Equivalently, $n$ is composite if it can be written as $n = st$, where $s$ and $t$ are integers and $1 < s,t < n$.
\end{definition}

\begin{lemma} \label{modprime}
	Let $a$, $b$ and $c$ be integers, and let $p$ be a prime.
	\begin{enumerate}
		\item If $p \nmid a$, then $\gcd(p, a) = 1$.
		\item If $a \mid bc$ and $\gcd(a, b) = 1$, then $a \mid c$.
		\item If $p \mid bc$, then $p \mid b$ or $p \mid c$.
	\end{enumerate}
\end{lemma}

\begin{proposition}[Modular Cancellation Law] \label{modcancel}
	Le $a, b, k$ and $m$ be integers, with $k \neq 0$. If $ak \equiv bk \pmod m$ and $\gcd(k,m) = 1$, then $a \equiv b \pmod m$.
\end{proposition}

\begin{scratch}
	Here, we have the following:
	\begin{align}
		ak                 & \equiv bk \pmod m \\
		\implies (ak - bk) & = mq              \\
		\implies (a-b)k    & = mq              \\
		\implies k         & \mid mq
	\end{align}
	By Lemma \ref{modprime}(2), since $\gcd(k,m)=1$, we get $k \mid q$, therefore, $q = kl$ for some integer $l$.
	$$(a-b)k = mkl \implies a-b = ml \implies m \mid (a-b) \implies a \equiv b \pmod m$$
\end{scratch}

\begin{proof}
	Since $ak \equiv bk \pmod m$, by Definition \ref{mod} and \ref{div}, we have, $ak - bk = mp$ for some integer $p$.
	$$ak-bk = mp \implies (a-b)k = mp \implies k \mid mp$$
	Now, since $\gcd(k,m) = 1$, by Lemma \ref{modprime}(2), we get $k \mid p \implies p = kq$ for some integer $q$.
	$$(a-b)k = mp \implies (a-b)k = mkq \implies (a-b) = mq \implies m \mid (a-b)$$
	Now, by Definition \ref{mod}, we can say that $a \equiv b \pmod m$.

	Hence, proved.
\end{proof}


\begin{theorem}[Fermat's Little Theorem]
	If $a$ is an integer and $p$ is a prime which does not divide $a$, then
	$$a^{p-1} \equiv 1 \pmod p.$$
\end{theorem}

\begin{scratch}
	The all-important observation is the following, which we explain through the example $a = 4$ and $p = 7$. Consider two sets:
	$$\{a, 2a, 3a, 4a, 5a, 6a\} \text{ and } \{1, 2, 3, 4, 5, 6\}$$
	In this example, since $a = 4$, this is the same as
	$$\{4, 8, 12, 16, 20, 24\} \text{ and } \{1, 2, 3, 4, 5, 6\}$$
	These look like completely different sets. But look what happens when you consider each of the numbers module $p$; the second set stays the same but the numbers in the first set change.
	$$\{4, 1, 5, 2, 6, 3\} \text{ and } \{1, 2, 3, 4, 5, 6\}$$
	Now, these two sets are the same. Since the order doesn't matter in multiplication, this means that
	$$a \cdot 2a \cdot 3a \cdot 4a \cdot 5a \cdot 6a \equiv 1 \cdot 2 \cdot 3 \cdot 4 \cdot 5 \cdot 6 \pmod 7$$
\end{scratch}

\begin{proof}
	Assume that $a$ is an integer and $p$ is prime which does not divide $a$. We begin by proving that when taking module $p$
	$$\{a, 2a, 3a, ..., (p-1)a\} \equiv \{1, 2, 3, ..., p-1\} \pmod p$$

	To do this observe that set on the right has every module except $0$. Thus, if we can show that no number on the left hand side is $0$ module $p$ and all of them are unique module $p$, then both sets must have the same elements module $p$.

	\underline{Step 1.} No element in the set $\{a, 2a, 3a, ..., (p-1)a\}$ is congruent to $0$ module $p$.

	Let us take an element $ia$ from the set and assume that it is congruent to $0$ modulo $p$. By Definition \ref{mod}, we get that $p \mid ia$.

	Since $p \nmid a$, by Lemma \ref{modprime}, we get that $\gcd(p, a) = 1$.

	By Lemma \ref{modprime}(2), we get that $p \mid i$. This is a contradiction since for all $i \in \{1, 2, 3, ..., p-1\}$, $p \nmid i$.

	Thus, our initial assumption that $ia \equiv 0 \pmod p$ must be wrong.

	Hence, proved.

	\bigbreak

	\underline{Step 2.} All elements in the set $\{a, 2a, 3a, ..., (p-1)a$ are unique modulo $p$.

	Let us take two elements $ia$ and $ja$ from the above set such that $ia \equiv ja \pmod p$.

	Since $\gcd(a, p) = 1$ as shown before, by Proposition \ref{modcancel}, we get that $i \equiv j \pmod p$.

	Since $i$ and $j$ are in the set $\{1, 2, 3, ..., (p-1)\}$, we can say that $i = j$ since no two elements in the set are congruent to each other module $p$.

	Thus, we get that $ia \equiv ja \pmod p \implies i = j$ and that all the elements are unique in the given set module $p$.


	These two steps complete the proof that
	$$\{a, 2a, 3a, ..., (p-1)a\} \equiv \{1, 2, 3, ..., p-1\} \pmod p$$

	Now, since the order doesn't matter in multiplication we can say that
	$$a \cdot 2a \cdot 3a \cdot ... \cdot (p-1)a \equiv 1 \cdot 2 \cdot 3 \cdot ... \cdot (p-1) \pmod p$$

	Since for each $i$ such that $2 \leq i \leq p-1$, we know that $p \nmid i$, we get that $\gcd(p, i) = 1$. Therefore, by Proposition \ref{modcancel}, we get
	$$\underbrace{a \cdot a \cdot a \cdot ... \cdot a}_{p-1 \text{ times}} \equiv 1 \pmod p$$
	$$ \implies a^{p-1} \equiv 1 \pmod p$$

	Hence, proved.
\end{proof}

\begin{theorem}[Euler's Theorem]
	If $a$ and $N$ are positive integers which are relatively prime then
	$$a^{\phi(N)} \equiv 1 \pmod{N}$$
\end{theorem}
