\chapter{Direct Proofs}

\begin{definition}
	A nonzero integer $a$ is said to \emph{divide} an integer $b$ if $b = ak$ for some integer $k$. When $a$ divides $b$, we write $a \mid b$ and when $a$ does not divide $b$, we write $a \nmid b$.
\end{definition}

\begin{theorem}[The Division Algorithm]
	For all integers $a$ and $m$ witht $m > 0$, there exist unique integers $q$ and $r$ such that 
	$$a = mq+r$$
	where $0 \leq r < m$.
\end{theorem}

\begin{definition}
	Let $a$ and $b$ be integers. If $c \mid a$ and $c \mid b$, then $c$ is said to be a \emph{common divisor} of $a$ and $b$.

	The \emph{greatest common divisor} of $a$ and $b$ is the largest $d$ such that $d \mid a$ and $d \mid b$. This number is denoted $\gcd(a,b)$.
\end{definition}
