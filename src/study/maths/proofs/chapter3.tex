\chapter{Sets}

\begin{definition}\label{set}
	A \emph{set} is ann unordered collection of distinct objects, which are called \emph{elements}.
	If $x$ is an element of a set $S$, we write $x \in S$. This is read "$x$ in $S$".
\end{definition}

\begin{definition}
	Common Sets:

	\begin{itemize}
		\item The set of \emph{natural numbers}, denoted $\mb{N}$, is the set $\{1, 2, 3, ...\}$.
		\item The set of \emph{integers}, denoted $\mb{Z}$, is the set $\{..., -3, -2, -1, 0, 1, 2, 3, ...\}$.
		\item The set without any elements, denoted $\phi$ or $\{\}$, is called the \emph{empty set}.
	\end{itemize}
\end{definition}

\begin{definition}\label{rational}
	The set $$\mb{Q} = \{ \frac{a}{b} : a, b \in \mb{Z}, b \neq 0 \}$$ is called the set of \emph{rational numbers}.
\end{definition}


The set of real numbers, denoted $\mb{R}$, is more difficult to define, so for now rely on your intuition. (Note to Self: We will define this in Real Analysis. )

\begin{definition}\label{subset}
	Suppose $A$ and $B$ are sets. If every element in $A$ is also an element of $B$, then $A$ is \emph{subset} of $B$, which is denoted $A \subseteq B$.
\end{definition}


\emph{Strategy To Prove $A \subseteq B$}: We can start with some element $x \in A$ and the condition for $A$. Now, we can apply logic and reasoning to show that this is the same as condition for $B$. Therefore, $x \in B$. 

Since we chose a arbitrary element of $A$, therefore, this is true for all elements of $A$. Hence, proved.
Moreover, we are not allowed to assume anything about $x$ beyond that it is in $A$. This is the reason that we can say that since it applies to an \emph{arbitrary} element of $A$, therefore, it applies to every element of $A$.


Notice that if $A = B$ then $A \subseteq B$. In the case, $A \subseteq B$ and $A \neq B$, we say that $A$ is a \emph{proper subset} of $B$, denoted by $A \subset B$. But we will not use this notation in this text.

\break
\emph{Strategy To Prove $A = B$}: To prove this, you will have to show that:
\begin{enumerate}
	\item Every element of $A$ is also in $B$ which means $A \subseteq B$.
	\item Every element of $B$ is also in $A$ which means $B \subseteq A$.
\end{enumerate}

\begin{definition}
	Set Operations:
	\begin{itemize}
		\item The \emph{union} of sets $A$ and $B$ is the set $A \cup B = \{ x : x \in A \text{ or } x \in B \}$.

		\item The \emph{intersection} of sets $A$ and $B$ is the set $A \cap B = \{ x : x \in A \text{ and } x \in B \}$.

		\item Likewise, if $A_1, A_2, A_3, ..., A_n$ are all sets, then the union of all of them is the set 
			$A_1 \cup A_2 \cup A_3 \cup ... \cup A_n = \{ x : x \in A_i \text{ for some } i \}$. 
			This set is also denoted as $$\bigcup_{i=1}^{n} A_i$$

		\item Likewise, if $A_1, A_2, A_3, ..., A_n$ are all sets, then the intersection of all of them is the set 
			$A_1 \cap A_2 \cap A_3 \cap ... \cap A_n = \{ x : x \in A_i \text{ for all } i \}$. 
			This set is also denoted as $$\bigcap_{i=1}^{n} A_i$$
	\end{itemize}
\end{definition}

\begin{definition}[Subtraction and Complements]\label{subtract}\label{complement}
Assume $A$ and $B$ are sets and $x \not\in B$means that $x$ is not an element of $B$.
\begin{itemize}
	\item The \emph{subtraction} of $B$ from $A$ is $A \setminus B = \{ x : x \in A \text{ and } x \not\in B \}$.
	\item If $A \subseteq U$, then $U$ is called a \emph{universal set} of $A$. The \emph{complement} of $A$ in $U$ is $A^c = U \setminus A$.
\end{itemize}
\end{definition}

\begin{definition}[Power Sets and Cardinality]\label{powerset}\label{cardinality}
	Assume $A$ is a set:
	\begin{itemize}
		\item The \emph{power set} of $A$ is $\mc{P}(A) = \{ X : X \subseteq A \}$.
		\item The \emph{cardinality} of $A$ is the number of elements ion $A$, and is denoted $|A|$.
	\end{itemize}
\end{definition}


\begin{definition}[Cartesian Product]\label{cartesian}
	Assume $A$ and $B$ are sets. The \emph{Cartesian Product} of $A$ and $B$ is $A \times B = \{(a, b) : a \in A \text{ and } b \in B \}$.
\end{definition}

\begin{proposition}
	Suppose $A$ and $B$ are sets. If $\mc{P}(A) \subseteq \mc{P}(B)$, then $A \subseteq B$.
\end{proposition}
\begin{proof}
	Assume $x \in A$ be an arbitrary element of $A$.

	Then by Definition \ref{powerset}, $\{ x \} \subseteq A$ and $\{ x \} \in \mc{P}(A)$. 
	Since $\mc{P}(A) \subseteq \mc{P}(B)$, by Definition \ref{subset}, we get that $\{ x \} \in \mc{P}(B)$. 
	Now, by Definition \ref{powerset}, we get that $\{ x \} \subseteq B$.
	And finally, by Definition \ref{subset}, we get that $x \in B$.

	Hence, proved that any arbitrary element of $A$ is also in $B$. Therefore, $A \subseteq B$.
\end{proof}


\begin{theorem}[De Morgan's Laws]\label{demorgan}
	Suppose $A$ and $B$ are subsets of a universal set $U$. Then, 
	$$(A \cup B)^c = A^c \cap B^c \text{ and } (A \cap B)^c = A^c \cup B^c$$
\end{theorem}

\section{Topology}

In topology, the sphere, the cube and the long snake are considered topologically equivalent. In fact, any other shape that you are able to knead the play-dough into would also be considered topologically equivalent, provided you follow these rules:
\begin{enumerate}
	\item You are not allowed to rip the play-dough into two pieces or poke any holes into it.
	\item You are not allowed to make any new connections that cannot be obtained with simple kneading.
\end{enumerate}

For example, it would be impossible to create a torus ( doughnut shape ) with your sphere shaped play-dough.

Topologists talk about \emph{genus} of a surface, where if a shape has $n$ holes in it,then it has genus $n$. If two ( orientable\footnote{This is a technical term to make this statement \emph{correct}.} ) surfaces have the same genuses, then one can be legally transformed into the other.

Leonard Euler discovered an important invariant for convex polyhedra. In his honour, it was named \emph{Euler Characteristic}.

What he noticed and proved was that given \emph{any} convex polyhedron, $V - E + F = 2$ where $V$ is number of vertices, $E$ is the number of edges and $F$ is the number of faces.

Notice that the formula starts with the $0$-dimensional object, then subtracts the $1$-dimensional object, then adds the $2$-dimensional object.

In a more general topological space, called a \emph{CW-Complex}, if $k_n$ is the number of $n$-diemensional "cells" in the CW-Complex, then the alternating sum
$$k_0 - k_1 + k_2 - k_3 + k_4 ... $$
is always a fixed constant. This constant is the topological invariant called the Euler Characteristic.

\begin{definition}\label{opensets}
	A set $V \subseteq \mb{R}$ is \emph{open} if every point in $V$ has an open interval around it that is also contained entirely in $V$.
	That is, for every, $x \in V$, there is a $\delta > 0$ such that the open interval $(x - \delta, x + \delta) \subseteq V$.
\end{definition}

\begin{theorem}
	Asssume all of the following sets are subsets of $\mb{R}$, and "open" is as  defined in Definition \ref{opensets}.
	\begin{enumerate}
		\item If $\{ V_{\alpha} \}$ is a family of open sets, then $\displaystyle \bigcup_{\alpha} V_{\alpha}$ is also an open set.


		\item If $\{V_1, V_2, ..., V_n\}$ is a family of finitely many open sets, then $\displaystyle \bigcap_{k=1}^n V_k$ is also an open set.
	\end{enumerate}
\end{theorem}


\begin{definition} 
	Let $X$ be a set. A \emph{topology} on $X$ is a family $\mc{T}$ of subsets of $X$ having the following properties:
	\begin{enumerate}
		\item $\phi$ and $X$ are in $\mc{T}$;

		\item The union of the sets in any subfamily of $\mc{T}$ is also in $\mc{T}$; and

		\item The intersection of the sets in any finite subfamily of $\mc{T}$ is also in $\mc{T}$.

	\end{enumerate}

\end{definition}


\subsection*{Manifolds}

Initially, topology is based on a lot of set theory which is called \emph{point-set topology}. After this, you may study \emph{algebraic topology} which will feel \emph{very} different than the point-set topology. And then \emph{geometric topology} fells different than both.

Here, we will talk a little about geometric topology.

One important are of topology is the study of so-called \emph{manifolds}, which are topological spaces with a few more bells and whistles.

\begin{itemize}
	\item If the topological space locally resembles real $n$-dimensional Euclidean space then it is called a \emph{topological manifold}.
	\item If that topological manifold is also equipped with a differentiable structure allowing one to do calculus on the manifold just like we do calculus on $\mb{R}$, then it is called a \emph{differentiable manifold}.
	\item If the differentiable manifold is also equipped with a function that allows you to determine distance and angles ( and hence notions of area, volume and "hyper-volume" ), then it is called a \emph{Reimannian manifold}.
\end{itemize}
