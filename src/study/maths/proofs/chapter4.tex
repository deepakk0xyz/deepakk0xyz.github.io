\chapter{Induction}

\begin{principle}[Induction]
	Consider a sequence of mathematical statements, $S_1, S_2, S_3, ...$
	\begin{itemize}
		\item Suppose $S_1$ is true; and
		\item Suppose, for each $k \in \mb{N}$, if $S_k$ is true, then $S_{k+1}$ is true.
	\end{itemize}

	Then, $S_n$ is true for every $n \in \mb{N}$.
\end{principle}

\begin{proposition}\label{triangular}
	For any, $n \in \mb{N}$,
	$$1 + 2 + 3 + ... + n = \frac{n(n+1)}{2}$$
\end{proposition}
\begin{proof}
	\underline{Base Case.} The base case is when $n = 1$, and
	$$1 =  \frac{1(1+1)}{2}$$
	as desired.

	\underline{Inductive Hypothesis.} Let $k \in \mb{N}$, and assume that 
	$$1 + 2 + 3 + ... + k = \frac{k(k+1)}{2}$$

	\underline{Induction Step.} We aim to prove that the result holds for $k+1$. That is, we wish to show that
	$$1 + 2 + 3 + ... + (k+1) = \frac{(k+1)((k+1)+1)}{2}$$

	To do this, we can write the sum as
	\begin{align*}
		1 + 2 + 3 + ... + k + (k+1) &= \frac{k(k+1)}{2} + (k+1) \\
																&= \frac{k(k+1)}{2} + \frac{2(k+1)}{2} \\
																&= \frac{k^2 + k + 2k + 2}{2} \\
																&= \frac{(k+1)(k+2)}{2}
	\end{align*}
	
	Hence, by induction, proved that $$1+2+3+...+n = \frac{n(n+1)}{2}$$ for all $n \in \mb{N}$.
\end{proof}

\begin{proposition}
	Let $T_n$ be the sum of first $n$ natural numbers. Then, for any $n \in \mb{N}$, $$T_n + T_{n+1} = (n+1)^2$$
\end{proposition}
\begin{proof}
	\underline{Base Case.} For $n = 1$, we have $T_1 = 1$ and $T_2 = 3$. 
	Therefore,
	$$T_1 + T_2 = 1 + 3 = 4 = 2^2 = (1+1)^2$$
	as desired

	\underline{Induction Hypopthesis.} Let $k \in \mb{N}$, and assume that
	$$T_k + T_{k+1} = (k+1)^2$$

	\underline{Induction Step.} Now, we wish to show that
	$$T_{k+1} + T_{k+2} = (k+2)^2$$

	To show this, we know that $T_{k+1} = T_k + (k+1)$. Thus, we can show that
	\begin{align*}
		T_{k+1} + T_{k+2} &= (T_k + (k+1)) + (T_{k+1} + (k+2)) \\
											&= T_k + T_{k+1} + 2k+3 \\
											&= (k+1)^2 + 2k + 3 \\
											&= k^2 + 2k + 1 + 2k + 3 \\
											&= k^2 + 4k + 4 \\
											&= (k+2)^2
	\end{align*}
	
	Hence, by induction, proved.
\end{proof}

\begin{proposition}
	For every $n \in \mb{N}$, the product of first $n$ odd natural numbers equals $\frac{(2n)!}{2^n \cdot n!}$. That is,
	$$1 \cdot 3 \cdot 5 \cdot ... \cdot (2n-1) = \frac{(2n)!}{2^n \cdot n!}$$
\end{proposition}
\begin{proof}
	\underline{Base Case.} For $n = 1$, we have
	$$1 = \frac{2!}{2^1 \cdot 1!} = \frac{2}{2} = 1$$
	as desired.

	\underline{Induction Hypothesis.} Let $k \in \mb{N}$ and assume that,
	$$1 \cdot 3 \cdot 5 \cdot ... \cdot (2k-1) = \frac{(2k)!}{2^k \cdot k!}$$

	\underline{Induction Step.} For $k+1$, we can show that,
	\begin{align*}
		1 \cdot 3 \cdot 5 \cdot ... \cdot (2k-1) \cdot (2(k+1)-1)
			&= \frac{(2k)!}{2^k \cdot k!} \cdot (2(k+1)-1) \\
			&= \frac{(2k)! \cdot (2k+1)}{2^k \cdot k!} \\
			&= \frac{(2k+1)!}{2^k \cdot k!} \cdot \frac{2k+2}{2k+2} \\
			&= \frac{(2k+2)!}{2^k \cdot k! \cdot 2(k+1)} \\
			&= \frac{(2(k+1))!}{2^{k+1} \cdot (k+1)!}
	\end{align*}

	Hence, by Induction, proved.
\end{proof}


\begin{principle}[Strong Induction]
	Consider a sequence of mathematical statements, $S_1, S_2, S_3, ...$
	\begin{itemize}
		\item Suppose $S_1$ is true, and 
		\item Suppose for any $k \in \mb{N}$, if $S_1, S_2, ..., S_k$ is true, then $S_{k+1}$ is true.
	\end{itemize}

	Then $S_n$ is true for every $n \in \mb{N}$.
\end{principle}

\begin{theorem}[Fundamental Theorem of Arithmetic]
	Every integer $n \geq 2$ is either prime or a product of primes.
\end{theorem}

\begin{proof}
	We prove this by Strong Induction.

	\underline{Base Case.} For $n = 2$, $2$ is prime so we are done.

	\underline{Inductive Hypothesis.} Let $k \in \mb{N}$ and $k \geq 2$ and assume that $2, 3, 4, ..., k$ satisfy the theorem, that is, all of them are either prime or product of primes.
	
	\underline{Induction Step.} Let us consider $k+1$ which is either prime or composite. Thus, there are two cases:

	\underline{Case 1.} $k+1$ is prime. Since it already satisfy the theorem. We are done.

	\underline{Case 2.} $k+1$ is composite.

	Now, since $k+1$ is composite, it can be written as $k+1 = st$ where $1 < s, t < k+1$. Since all numbers $2, 3, 4, ..., k$ satisfy the theorem, then $s$ and $t$ are also either prime or product of primes.

	Thus, we can say $s = p_1 \cdot p_2 \cdot ... \cdot p_m$ and $t = q_1 \cdot q_2 \cdot ... \cdot q_l$ where $p_i$ and $q_i$ are primes.

	If $s$ or $t$ are prime, then we will only have a single prime in their product expansion.

	Now, we can show
	$$k+1 = st = (p_1 \cdot p_2 \cdot ... \cdot p_m) \cdot (q_1 \cdot q_2 \cdot ... \cdot q_l)$$

	Thus, $k+1$ is written as a product of primes.

	\underline{Conclusion.} By strong induction, every positive integer larger than $2$ can be written as a product of primes.
\end{proof}

\section*{Notes}

\begin{itemize}
	\item If a theorem holds for every $n \in \mb{Z}$, then you may have to perform two inductions: One for the positive values and one for the negative values.

		For instance, if you prove that $S_0$ holds, and you prove that $S_k \implies S_{k+1}$ for every $k \in \{0,1,2,3,...\}$, and you prove that $S_k \implies S_{k-1}$ for every $k \in \{0,-1,-2,-3,...\}$, then combined this would prove thata $S_n$ holds for every $n \in \mb{Z}$.

	\item It turns out that this "backwards" technique does not work only with negeative numbers, it also provides a second way to prove the positive case. 

		Here's how: Suppose you wish to prove a result for every $n \in \mb{N}$. If you can prove that it is true for an infinite sequence of base cases, like for the $n \in \{1, 2, 4, 8, 16, 32, ...\}$, and you can prove that for every $k \in \mb{N}$ with $k \geq 2$, that $S_k \implies S_{k-1}$, then every case must hold.

		For example, why is the $n = 60$ case true? Well, the $n = 64$ case was one of the base cases that was proven, and therefore by backwards induction, $S_64 \implies S_63 \implies S_62 \implies S_61 \implies S_60$, which shows that the $n = 60$ case holds.

		As long as you can show some infinite sequence works, then you can backwards induct to any case. This technique is called \emph{backwards induction}.


	\item Supppose you want to prove something only for all $n \in \{1, 2, 3, ..., 100\}$, you can still use induction. You base case would be $n = 1$ and inductive hypothesis will assume $k \in \{1,2,3,...,99\}$, and then in your induction step you would show that $S_k \implies S_{k+1}$. In this way, induction can also be used to prove a result in finitely many cases.
\end{itemize}

\bp Use induction or strong induction to prove the following for all $n \in \mb{n}$.
\begin{enumerate}
	\item 
\end{enumerate}
\ep
