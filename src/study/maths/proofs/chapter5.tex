\chapter{Logic}

Logic is the process of deducing information correctly -- it is \emph{not} the proces of deducing correct information.

If the logic is valid \emph{and} the statements are true, then it is called \emph{sound}.


\begin{definition}
	A \emph{statement} is a sentence or mathemetical expression that is either true or false.
\end{definition}

A related notion is that of an \emph{open statement}, which are sentences of mathematical expressions which (1) do not have a truth value, (2) depend on some unknown, like a variable $x$ or an arbitrary function $f$, and (3) when the unknown is specified, then the open sentence becomes a statement ( and so has a truth value ). Their truth value depends on which value of $x$ or $f$ one chooses.

\begin{example}
	Here are four examples of open sentences:
	\bnum
	\item $x + 7 = 12$
	\item $f$ is continuous.
	\item $3 \mid x$
	\item $x$ is even.
	\enum
\end{example}

\begin{named}[Notation]
	Let $P$ and $Q$ be statements or open statements.
\bnum
	\item $P \land Q$ means "$P$ and $Q$".
	\item $P \lor Q$ means "$P$ or $Q$".
	\item $\lnot P$ means "not $P$".
	\item $P \implies Q$ means "$P$ implies $Q$".
	\item $P \iff Q$ means "$P$ if and only if $Q$".
\enum
\end{named}

\begin{definition}
	The \emph{converse} of $P \implies Q$ is $Q \implies P$.
\end{definition}

\begin{theorem}[De Morgan's Logic Laws]
	If $P$ and $Q$ are statements, then
	$$\lnot (P \land Q) \iff \lnot P \lor \lnot Q \quad \text{and} \quad
		\lnot (P \lor Q) \iff \lnot P \land \lnot Q$$
\end{theorem}

\begin{named}[Note]
We can say that
\bitem
\item $\lnot \land = \lor$
\item $\lnot \lor = \land$
\item $\lnot \forall = \exists$
\item $\lnot \exists = \forall$
\eitem
\end{named}

From the truth table, we have 
$$\lnot (P \implies Q) \implies P \land \lnot Q$$

\begin{definition}
	The \emph{contrapositive} of $P \implies Q$ is $\lnot Q \implies \lnot P$.
\end{definition}

\begin{theorem}
	An implication is equivalent to its contrapositive. That is,
	$$(P \implies Q) \iff (\lnot Q \implies \lnot P)$$
\end{theorem}


\section*{Introduction to Real Analysis}

This is a miscellaneous section.

\begin{definition}
	A sequence $a_1, a_2, a_3, \ldots$ \emph{converges} to $a \in \mb{R}$ if for all $\varepsilon > 0$ there exists some $N$ such that $|a_n - a| < \varepsilon$ for all $n > N$.
\end{definition}

\begin{definition}
	Let $f : \mb{R} \rightarrow \mb{R}$. We say that $f$ is continuous at $c$ if for all $\varepsilon >0$ there exists some $\delta > 0$ such that for every $x \in \mb{R}$ for which $0 < |x - c| < \delta$, we have
	$$|f(x)-f(c)| < \varepsilon$$
\end{definition}

