\section*{Exercises}

\begin{problem}
  Read \emph{The Secret of Raising Smart Kids} by Carol Dweck and write a few paragraphs about what you learned and how it may help you be successful in proof-based math class.
\end{problem}

\begin{solution}
Not Interested.
\end{solution}

\begin{problem}
  Explain the error in the following "proof" that $2 = 1$. \\
  Let $x = y$. Then,

\begin{align}
x^2 &= xy \\
x^2 - y^2 &= xy - y^2 \\
(x+y)(x-y) &= y(x-y) \\
x+y &= y \\
2y &= y \\
2 &= 1
\end{align}
\end{problem}

\begin{solution}
  Since $x = y$, $x-y = 0$ and therefore, we cannot divided by $x-y$ in step $3$ to get $x+y = y$ from $(x+y)(x-y) = y(x-y)$. Thus, solved.
\end{solution}


\begin{problem}
  Suppose that $m$ and $n$ are positive odd integers. Using $2 \times 1$ dominos,

  (a) Does there exist a perfect cover of the $m \times n$ chessboard?

  (b) If I remove 1 square from the $m \times n$ chessboard, will it have a perfect cover?
\end{problem}

\begin{solution}[a]
  In this case, there are $m \times n$ cells on the board which is an odd number. Since each domino covers only $2$ cells, the total number of cells covered will always be even.

  Hence, no perfect cover exists.
\end{solution}

\begin{scratch}[b]
  Let us take $3 \times 3$ chessboard. There are $9$ cells on the board. Without loss of generality, let us say there are $4$ white cells and $5$ black cells.

  Since a domino always covers $1$ white and $1$ black cell, the number of white and black cell must be equal for a perfect cover.

  Let us remove a black cell from the above chessboard. Now there are $4$ white cells and $4$ black cells.

  Checking all $5$ black squares for removal, we find that we have a cover in every case.
\end{scratch}

\begin{solution}[b]
  Let us assume that the board has $x$ white cells and $x+1$ black cells. We can show that in this case all the corners must be black.

  Since each domino must cover only $1$ white and $1$ black cell, we can start with a covering where all dominos are put horizontally starting from the left. \\
  In this case, we will cover the entire chessboard except the last column.

  For the last column, we can put the dominos vertically starting from the top, but since it has an odd number of cells, there will be the bottom right corner.

  We can remove the remaining cell and get a perfect cover of the chess board.

  Hence, proved.
\end{solution}

\begin{problem}
  The game \emph{Tetris} is played with five different shapes
\end{problem}