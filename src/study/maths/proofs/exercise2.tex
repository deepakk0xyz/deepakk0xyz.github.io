\section*{Exercises}

\begin{problem}
	List $5$ skills that are important for someone to be successful in a college math class. Which skills seem most important for an upper-division math class? Which skills do you want to work to improve?
\end{problem}
\begin{solution}Not Interested.\end{solution}

\begin{problem}
	The following are the squares of four numbers, each ending in $5$.
	$$15^2 = 225, 25^2 = 625, 35^2 = 1225, 45^2 = 2025$$
	Lookiung at these four squares, do you see anythign interesting about their answers. Once you have noticed a pattern, answer the following.
	\bigbreak
	(a) Write down a conjecture that explains that the answer is for the square of any integer ending in $5$.

	\begin{solution}
		\begin{named}[Conjecture]
			Any integer ending in $5$ has a square which ends in $25$.
		\end{named}
	\end{solution}

	(b) Give four more examples illustrating your conjecture.
	\begin{solution}
		$$5^2 = 25, 55^2 = 3025, (-15)^2 = 225, (-25)^2 = 625$$
	\end{solution}

	(c) Prove your conjecture.
	\begin{solution}
		For any integer ending in $5$ would be of the form $10a+5$ for some integer $a$. Now, we can show that,
		\begin{align}
			(10a+5)^2 &= (10a)^2 + 2 \cdot 10a \cdot 5 + 5^2 \\
				&= 100a^2 + 100a + 25 \\
				&= 100(a^2+a) + 25
		\end{align}
		Since $a$ is an integer, therefore, $a^2 + a$ is also an integer.
		Thus, the square is of the form $100k+25$ where $k = a^2+a$ is an integer.

		This proves that the square of the number ends in $25$.

		Hence, proved.
	\end{solution}
\end{problem}

\begin{problem}
	For each of the following, prove that it is true.

	\bigbreak
	(a) The sum of an even integer and an odd integer is odd.

	\begin{solution}
		Let $a$ be an even integer and $b$ be an odd integer. By definition, there exist some integers $k$ and $l$ such that $a = 2k$ and $b = 2l+1$.

		Now, 
		$$a + b = 2k + 2l + 1 = 2(k+l) + 1$$
		Since $k$ and $l$ are integers, therefore, $k+l$ is also an integer. Thus, for $a+b$ there exist an integer $m=k+l$ such that $a+b = 2m+1$.
		This proves that $a+b$ is odd.

		Hence, proved.
	\end{solution}

	(b) The product of two even integers is even.

	\begin{solution}
		Let $a = 2k$ and $b = 2l$ be two even integers such that $k$ and $l$ are integers. Now, 
		$$a \cdot b = 2k \cdot 2l = 4kl = 2 \cdot ( 2kl )$$
		Since $k$ and $l$ are integers, therefore, $2kl$ is also an integers. This proves that $a \cdot b = 2m$ for some integer $m$. 
		Hence, proved.
	\end{solution}

	(c) The product of two odd integers is odd.
	\begin{solution}
		Let $a = 2k+1$ and $b = 2l+1$ be two odd integers for some integers $k$ and $l$. Now,
		\begin{align}
			a \cdot b &= (2k+1) \cdot (2l + 1) \\
								&= 2k \cdot 2l + 2k \cdot 1 + 1 \cdot 2l + 1 \cdot 1 \\
								&= 4kl + 2k + 2l + 1 \\
								&= 2(2kl + k + l) + 1
		\end{align}
		Here, $2kl+k+l$ is an integer since $k$ and $l$ are integers. This proves that $a \cdot b = 2m+1$ for some integer $m$ and thus, $a \cdot b$ is odd.

		Hence, proved.
	\end{solution}

	(d) The product of an even integer and an odd integer is even.
	\begin{solution}
		Let $a = 2k$ and $b = 2l+1$ be an even and odd integer respectively for some integers $k$ and $l$. Now,
		$$ a \cdot b = 2k \cdot b = 2 \cdot (kb)$$
		Here, since $k$ and $b$ are integers, we have shown that $a \cdot b = 2m$ for some integer $m$ and thus, $a \cdot b$ is even.

		Hence, proved.
	\end{solution}
	
	(e) An even integer squaredd is an even integer.
	\begin{solution}
		Let $a = 2k$ be an even integer for some integer $k$. Now,
		$$a^2 = a \cdot a = 2k \cdot 2k = 2 \cdot (2k^2)$$
		Here $2k^2$ is an integer since $k$ is an integer. Thus, we have shown that $a^2 = 2m$ for some integer $m$ and thus, $a^2$ is even.

		Hence, proved.
	\end{solution}

\end{problem}


\begin{problem}
	For each of the following, prove that it is true.
	\bigbreak
	(a) If $n$ is an even integer, then $-n$ is an even integer.
	\begin{solution}
		Let $n = 2k$ be an even integer for some integer $k$. Now, $-n = -2k = 2 \cdot (-k)$. Since $k$ is an integer, $-k$ is also an integer. Thus, we have shown that $-n$ is an even integer. Hence, proved.
	\end{solution}

	(b) If $n$ is an odd integer, then $-n$ is an odd integer.
	\begin{solution}
		Let $n = 2k+1$ be an odd integer for some integer $k$. Now, $-n = -(2k+1) =-2k-1 = -2k-2+1 = 2(-k-1) + 1$.

		Since $k$ is an integer, $-k-1$ is also an integer. Thus, we have shown that $-n$ is an odd integer. Hence, proved.
	\end{solution}

	(c) If $n$ is an even integer, then $(-1)^n = 1$.
	\begin{solution}
		Let $n = 2k$ be an even ingger for some integer $k$. Now, 
		$$(-1)^n = (-1)^{2k} = ((-1)^2)^k = 1^k = 1$$
		Hence, proved.
	\end{solution}
\end{problem}

\begin{problem}
	Prove the following:
	\bigbreak
	(a) If $n$ is odd, then $n^2 + 4n + 9$ is even.
	\begin{solution}
		Let $n = 2k+1$ be an odd integer for some integer $k$.
		$$n^2 = (2k+1)^2 = 4k^2 + 4k + 1$$
		$$4n = 4(2k+1) = 8k + 4$$
		$$n^2 + 4n + 9 = 4k^2 + 4k + 1 + 8k + 4 + 9 = 4k^2 + 12k + 10 = 2(2k^2+6k+5)$$
		Since $k$ is an integer, therefore, $2k^2+6k+5$ is also an integer. Thus, we have shown that $n^2 + 4n + 9 = 2m$ for some integer $m$ and therefore, $n^2 + 4n + 9$ is even.

		Hence, proved.
	\end{solution}
	
	(b) If $n$ is odd, then $n^3$ is odd.
	\begin{solution}
		Let $n = 2k+1$ for some integer $k$. Then,
		$$n^3 = (2k+1)^3 = (2k+1)(4k^2+4k+1)$$
		$$n^3 = 8k^3+8k^2+2k+4k^2+4k+1 = 8k^3 + 12k^2 + 6k + 1$$
		$$n^3 = 2(4k^3+6k^2+3k) + 1$$
		Hence, proved.
	\end{solution}

	(c) If $n$ is even, then $n+1$ is odd.
	\begin{solution}
		Let $n = 2k$ for some integer $k$, then $n+1 = 2k+1$ which is odd by definition. Hence, proved.
	\end{solution}
\end{problem}

\begin{problem}
	Prove the following. For each $m$ and $n$ are integers.
	\bigbreak
	(a) If $m$ and $n$ are odd, then $5m-3n$ is even.
	\begin{solution}
		Let $m = 2k+1$ and $n = 2l+1$ by definition of odd numbers.
		$$5m-3n = 5(2k+1)-3(2l+1) = 10k+5-6l-3 = 10k-6l+2 = 2(5k-3l+1)$$
		Hence, proved.
	\end{solution}

	(b) If $m$ and $n$ are even, then $3mn$ is divisible by $4$.
	\begin{solution}
		Let $m = 2k$ and $n = 2l$ by definition of even numbers.
		$$3mn = 3 \cdot 2k \cdot 2l = 12kl = 4 \cdot  3kl$$
		By Definition \ref{div}, $4 \mid 3mn$. Hence, proved.
	\end{solution}
\end{problem}

\begin{problem}Skipped.\end{problem}
\begin{problem}Skipped.\end{problem}
\begin{problem}Skipped.\end{problem}

\begin{problem}
	Prove the following. For each $m$, $n$ and $l$ are integers.
	\bigbreak
	(a) If $m \mid n$, then $m^2 \mid n^2$
	\begin{solution}
		By Definition \ref{div}, since $m \mid n$, there exists an integer $k$ such that $n = mk$ which implies that $n^2 = m^2 \cdot k^2$.
		Again by definition \ref{div}, $m^2 \mid n^2$. 
		Hence, proved.
	\end{solution}

	(b) If $m \mid n$, then $m \mid (7n^3 + 13n^2 - n)$
	\begin{solution}
		We can show that $7n^3 + 13n^2 - n = n \cdot (7n^2 + 13n - 1)$ which by definition \ref{div} implies that $n \mid (7n^3 + 13n^2 - n)$.
		Now, by Proposition \ref{transdiv}, since $n \mid n$, we can say that $m \mid (7n^3 + 13n^2 - n)$
	\end{solution}

	(c) If $m \mid n$ and $m \mid l$, then $m \mid (n+l)$.
	\begin{solution}
		By Definition \ref{div}, we get $n = ma$ and $l = mb$ for some integers $a$ and $b$.
		Therefore, $n+l = ma + mb = m(a+b)$ which by definition \ref{div} implies that $m \mid (n+l)$. 
		Hence, proved.
	\end{solution}

	(d) If $3 \mid 2n$, then $3 \mid n$
	\begin{solution}
		Since $\gcd(3,2) = 1$, then by Lemma \ref{modprime}(2), $3 \mid n$.
	\end{solution}

	(e) If $9 \mid 6n$, then $3 \mid n$.
	\begin{solution}
		By Definition \ref{div}, for some integer $a$, $6n = 9a \implies 2n = 3a$. Thus, by (d), $3 \mid n$.
	\end{solution}

	(f) If $m^3 \mid n$ and $n^4 \mid t$ then $m^12 \mid t$.
	\begin{solution}
		By Definition \ref{div}, $t = n^4 \cdot k$ and $n = m^3 \cdot l$ for some integers $k$ and $l$. This implies that $t = (m^3 \cdot l)^4 \cdot k = m^12  \cdot l^4 \cdot k$. Thus, by Definition \ref{div}, $m^12 \mid t$. Hence, proved.
	\end{solution}

\end{problem}


\begin{problem}Skipped.\end{problem}

\begin{problem}
	Prove that if $m$ and $n$ are positive real numbers and $m < n$, then $m^2 < n^2$. You may use the fact that if $a < b$ and $c$ is positive, then $ac < bc$.
\end{problem}

\begin{solution}
	\begin{align}
		m < n, m > 0 &\implies m^2 < mn \\
		m < n, n > 0 &\implies mn < n^2 \\
		m^2 < mn, mn < n^2 &\implies m^2 < n^2
	\end{align}
\end{solution}

\begin{problem}
	Define the absolute value of a real number $x$ in this way:
	$$ 
	|x| = 
	\begin{cases}
		x &\text{ if } x \geq 0 \\
		-x &\text{ if } x < 0
	\end{cases}
	$$
	Prove that $|xy| = |x| \cdot |y|$.
\end{problem}

\begin{solution}
	Let us take 4 cases:

	\underline{Case 1.} $x = 0$ or $y = 0$. 

	Without loss of generality, let us assume that $x = 0$.

	This implies that $xy = 0$. Thus, $|xy| = 0$ and $|x| = 0$, therefore, $|x|\cdot|y| = 0 = |xy|$. Hence, proved.

	\underline{Case 2.} $x > 0$ and $y > 0$.

	This implies that $xy > 0$ so $|xy| = xy, |x| = x$ and $|y| = y$.
	Thus, $|xy| = xy = |x|\cdot|y|$. Hence, proved.

	\underline{Case 3.} $x < 0$ and $y < 0$.

	This implies that $xy > 0$ so $|xy| = xy, |x| = -x$ and $|y| = -y$. 
	Thus, $|x|\cdot|y| = (-x) \cdot (-y) = xy = |xy|$. Hence, proved.

	\underline{Case 4.} Without loss of generality, $x < 0$ and $y > 0$.

	This implies that $xy < 0$. Here, $|xy| = -xy, |x| = -x$ and $|y| = y$.
	Therefore, $|x| \cdot |y| = (-x) \cdot y = -xy = |xy|$.
	Hence, proved.
\end{solution}

\begin{problem}
	Prove that if $m$, $n$ and $t$ are integers, then at least one of $m-n$, $n-t$ and $m-t$ is even. Also write down three example, and show which of $m-n$, $n-t$ or $m-t$ are even.
\end{problem}

\begin{scratch}
	$$m = 1, n = 2, t = 3 \implies m-n = -1, n-t = -1, m-t = -2 \implies m-t \text { is even.}$$
	$$m = 54, n = 29, t = 20 \implies m-n = 25, n-t = 9, m-t = 34 \implies m-t \text { is even.}$$
	$$m = 19, n = 45, t = 77 \implies m-n = -26, n-t = -32, m-t = -58 \implies m-n \text { is even.}$$
\end{scratch}

\begin{proof}
	Since all integers are either even or odd, by The Pigeonhole Principle ( \ref{pigeonhole} ), we know that atleast two of $m, n$ or $t$ have the same parity, i.e., two of them are either both odd or both even.

	\begin{lemma} \label{paritydiff}
		If $a$ and $b$ are integers that have the same parity then $a-b$ is even.
	\end{lemma}
	\begin{proof}[Lemma Proof]
		\underline{Case 1.} $a$ and $b$ are even.

		By definition, $a = 2k$ and $b = 2l$ then $a-b = 2k-2l = 2(k-l)$. Thus, $a-b$ is even by definition.

		\underline{Case 2.} $a$ and $b$ are odd.

		By definition, $a = 2k+1$ and $b = 2l+1$ then $a-b = (2k+1)-(2l+1) = 2k-2l = 2(k-l)$. Thus, $a-b$ is even by definition.

		Hence, proved.
	\end{proof}

	Since atleast two of $m, n$ or $t$ have the same parity then there is atleast one even number in $m-n$, $n-t$ and $m-t$ by Lemma \ref{paritydiff}.
\end{proof}


\begin{problem}
	Prove the following:
	\bigbreak
	(a) Prove that if $n$ is a positive integer, then $4$ divides $1 + (-1)^n(2n-1)$.
	\begin{solution}
		Given $n > 0$, we know that $n$ is either even or odd. Let us define two cases:

		\underline{Case 1.} $n$ is even.

		By Definition, $n = 2k$ for some integer $k$. 
		Therefore, we know that $(-1)^n = (-1)^{2k} = ((-1)^2)^k = (1)^k = 1$. Thus, 
		$$1 + (-1)^n(2n-1) = 1 + 2n-1 = 2n = 2 \cdot 2k = 4k$$
		Thus, by Defintion \ref{div}, $4 \mid 1 + (-1)^n(2n-1)$
		
		\underline{Case 2.} $n$ is odd.

		By Definition, $n = 2k+1$ for some integer $k$. 
		Therefore, we know that $(-1)^n = (-1)^{2k+1} = (-1)^{2k} \cdot (-1) = -1$ since $(-1)^{2k} = 1$.
		Thus, we get,
		\begin{align}
			1 + (-1)^n(2n-1) &= 1 + (-1)(2n-1) \\
											 &= 1 - 2n + 1 = 2 - 2n \\
											 &= 2 - 2(2k+1) \\
											 &= 2 - 4k - 2 = -4k
		\end{align}

		Thus, by Defintion \ref{div}, $4 \mid 1 + (-1)^n(2n-1)$


		Hence, proved.
	\end{solution}

	(b) Prove that every multiple of $4$ is equal to $1 + (-1)^n(2n-1)$ for some positive integer $n$.
	\begin{solution}
		Let us take a multiple of $4$ as $4k$. We have two cases:
		
		\underline{Case 1.} $k \geq 0$

		Since $(-1)^{2k} = 1$, we can show,
		$$4k = 2 \cdot 2k = 1 + 2 \cdot 2k - 1 = 1 + (-1)^{2k}(2 \cdot 2k - 1)$$
		If we substitute, $n = 2k$, we get, 
		$$4k = 1 + (-1)^n(2n-1)$$
		Thus, $k \geq 0$ satisfies the condition.

		\underline{Case 2.} $k < 0$
		Let us define $l = -k$ which implies that $l > 0$.
		$$4k = -4l =  2 - 4l - 2 = 2 - 2(2l+1)$$
		Let us substitute $n = 2l+1$.
		$$4k = 2 - 2n = 1 - 2n + 1 = 1 - (2n-1) = 1 + (-1)(2n-1)$$
		Since $(-1)^{2l} = 1 \implies (-1)^{2l+1} = (-1)^n = -1$, we get that
		$$4k = 1 + (-1)^n(2n-1)$$

		Thus, proved.
	\end{solution}
\end{problem}


\begin{problem}Skipped.\end{problem}
\begin{problem}Skipped.\end{problem}
\begin{problem}Skipped.\end{problem}

\begin{problem}
	Let $a$ and $b$ be positive integers, and suppose $r$ is the nonzero remainder when $b$ is divided by $a$. Prove that when $-b$ is divided by $a$, the remainder is $a-r$.
\end{problem}

\begin{solution}
	By Theorem \ref{divalgo}, we can say that $b = aq + r$ such that $0 \leq r < a$.
	But we are given that $r$ is nonzero, therefore, $0 < r < a$. Now,

	$$-b = -aq-r = -a-aq+a-r = a(-1-q) + (a-r)$$
	
	Since $0 < r < a$, we can show that 
	$$r < a \implies a - r > 0, r > 0 \implies r > a-a \implies a > a-r$$

	Thus, we have $0 < a-r < a$. By applying Theorem \ref{divalgo}, we get that the remainder on dividing $-b$ by $a$ is $a-r$.

	Hence, proved.

\end{solution}

\begin{problem}
	Determine the remainder when $3^302$ is divided by $28$, and show without a calculator how you found the answer.
\end{problem}

\begin{solution}
	\begin{align}
		3^{302} &= 3^{2} \cdot 3^{300} \pmod{28} \\
						&= 9 \cdot (3^3)^{100}  \pmod{28} \\
						&= 9 \cdot 27^{100} \pmod{28} \\
	\end{align}
	Since $27 \equiv -1 \pmod{28}$, then by Proposition \ref{modprop}(3), we get that $27^{100} \equiv (-1)^{100} \pmod{28}$ since exponentiation is just repeated multiplication.
	\begin{align}
		3^{302} &= 9 \cdot (-1)^{100} \pmod{28} \\
		3^{302} &= 9 \cdot 1 \pmod{28} \\
		3^{302} &= 9 \pmod{28}
	\end{align}

	Thus, the answer is $9$.

\end{solution}

\begin{problem}
	Assume that $a, b, c, d$ and $n$ are integers. Also assume that $a \equiv b \pmod{n}$ and $c \equiv d \pmod n$. Prove the following

	(i) $a-c \equiv b-d \pmod{n}$
	\begin{solution}
		By Definition \ref{mod}, we get that $n \mid (a-b)$ and $n \mid (c-d)$ which implies that $a-b = nk$ and $c-d = nl$ for some integers $k$ and $l$. Now,
		$$(a-c)-(b-d) = a-c-b+d = (a-b)-(c-d) = nk-nl = n(k-l)$$
		Thus, $n \mid ((a-c)-(b-d))$ and by Definition \ref{mod}, $a-c \equiv b-d \pmod{n}$
	\end{solution}

	(ii) $a \cdot c \equiv b \cdot d \pmod{n}$
	\begin{solution}
		As shown above, we have $a-b = nk$ and $c-d = nl$ for some integers $k$ and $l$. This gives us $a = b+nk$ and $c = d+nl$. Now,
		\begin{align}
			a \cdot c &= (b + nk) \cdot (d + nl) \\
								&= bd + bnl + dnk + n^2 kl \\
								&= bd + n(bl+dk+nkl) \\
			\implies (ac - bd) &= n(bl+dk+nkl) \\
			\implies n \mid (ac - bd)
		\end{align}
		Thus, by Definition \ref{mod}, $ac \equiv bd \pmod{n}$.
	\end{solution}
\end{problem}

\begin{problem}
	Assume that $a$ is an integer and $p$ and $q$ are distint primes. Prove that if $p \mid a$ and $q \mid a$, then $pq \mid a$.
\end{problem}

\begin{solution}
	By Definition \ref{div}, we get that $a = pk$ for some integer $k$. Now, $q \mid a \implies q \mid pk$.

	Since $p$ and $q$ are distinct primes their only factors are $1,p$ and $1,q$ respectively. Therefore, $\gcd(p, q) = 1$. Thus, by Lemma \ref{modprime}(2), we get $q \mid k$.

	Thus, by Definition \ref{div}, $k = ql$ for some integer $l$ and we get that $a = pql$ which by Definiton \ref{div} implies that $pq \mid a$.

	Hence, proved.
\end{solution}

\begin{problem}
Prove that if $abc$ is a multiple of $10$ then atleast one of $ab$, $bc$ or $ac$ is a multiple of $10$.
\end{problem}

\begin{solution}
	By Proposition \ref{transdiv}, since $2 \mid 10$ and $5 \mid 10$, we get that $2 \mid abc$ and $5 \mid abc$.

	Now, by Lemma \ref{modprime}(3), we get that atleast one of the following is true: $2 \mid a$ or $2 \mid b$ or $2 \mid c$ since $2$ is a prime.

	Similarly, since $5$ is a prime atleast one of these is true: $5 \mid a$ or $5 \mid b$ or $5 \mid c$.

	There are two cases:
	
	\underline{Case 1.} $2$ and $5$ divide the same number among $a, b$ or $c$.

	Without loss of generality, let us assume that $2 \mid a$ and $5 \mid a$. Then by previous problem, we can say that $10 \mid a$ which implies $a = 10k$ for some integer $k$.
	
	Since $a = 10k$, we have $ab = 10kb$ which by Definition \ref{div} means $10 \div ab$. 

	\underline{Case 2.} $2$ and $5$ divide different numbers among $a, b$ or $c$.

	Without loss of generality, let us assume that $2 \mid a$ and $5 \mid b$. Thus, by Definition \ref{div}, we have $a = 2k$ and $b = 5l$ for some integers $k$ and $l$.

	Now, $ab = 10kl$ which by Definition \ref{div} means $10 \mid ab$.

	Thus, in both cases there exists atleast one number among $ab, bc$ or $ac$ divisible by $10$.

	Hence, proved.
\end{solution}

\begin{problem}
	Assume that $a, b$ and $c$ are integers and $a^2 \mid b$ and $b^3 \mid c$. Prove that $a^6 \mid c$.
\end{problem}
\begin{solution}
	By Definition \ref{div}, we have $b = a^2 \cdot k$ and $c =  b^3 \cdot l$ for some integers $k$ and $l$. Thus,
	$$c = b^3 \cdot l = (a^2 \cdot k)^3 \cdot l = a^6 \cdot k^3 \cdot l$$
Hence, by Definiton \ref{div}, $a^6 \mid c$. Hence, proved.
\end{solution}

\begin{problem}
	Prove that for every integer $n$, either $n^2 \equiv 0 \pmod{4}$ or $n^2 \equiv 1 \pmod{4}$.
\end{problem}

\begin{scratch}
	$$n^2 = 4k \implies 4 \mid n^2 \implies 2 \mid n \implies n = 2l$$
	$$n^2 = 4k + 1 \implies 4 \mid (n^2-1) \mid 4 \mid (n-1)(n+1)$$
	Since $n-1$ and $n+1$ are $2$ apart, they are either both odd or both even.
	Since product of odd numbers is always odd, they must be even.
	Thus, $n$ is odd.

	Here, we see that if $n^2 \equiv 0 \pmod{4}$ then $n$ must be even. And otherwise $n$ must be odd.
\end{scratch}

\begin{solution}
	Let us take two cases:

	\underline{Case 1.} $n$ is even.

	By Definition, $n = 2k$ for some integer $k$. Thus, $n^2 = 4k^2$ which by Definition \ref{div} and \ref{mod} means that $4 \mid n^2 \implies n^2 \equiv 0 \pmod{4}$.

	\underline{Case 2.} $n$ is odd.

	By Definition, $n = 2k+1$ for some integer $k$. Thus, 
	$$n^2 = (2k+1)^2 = 4k^2 + 4k + 1 = 4(k^2+k) + 1$$
	$$\implies n^2 - 1 = 4(k^2+k)$$

	Thus, by Definition \ref{mod},  we get that $n^2 \equiv 1 \pmod{4}$.

	Hence, proved.
\end{solution}


\begin{problem}Skipped.\end{problem}

\begin{problem}
	The Pythagorean theorem involves integers $a, b$ and $c$ for which $a^2 + b^2 = c^2$. Prove that if three integers satisfy this relationship, then either $a$ or $b$ will be divisible by $3$.
\end{problem}

\begin{scratch}
	For every integer $n$, we know $n \pmod{3}$ is either $0,1$ or $2$.
	This implies that $n^2 \pmod{3}$ is either $0, 1$ or $4$ which is the same as $0$ or $1$.

	Thus, $a^2, b^2$ and $c^2$ are either $0$ or $1$ modulo $3$.

	Let us assume that both $a^2$ and $b^2$ are $1$ modulo $3$. Then, $c^2$ must be $2$ modulo $3$ which is a contradiction.
	Therefore, either $a^2$ or $b^2$ must be divisible by $3$ which implies either $a$ or $b$ must be divisible by $3$.
\end{scratch}

\begin{solution}
	\begin{lemma} \label{sqmod3}
		For any integer $n$, either $n^2 \equiv 0 \pmod{3}$ or $n^2 \equiv 0 \pmod{3}$.
	\end{lemma}
	\begin{proof} 
		For any integer $n$, we have $3$ cases:

		\underline{Case 1.} $n \equiv 0 \pmod 3 \implies n^2 \equiv 0 \pmod{3}$ by Proposition \ref{modprop}(3).

		\underline{Case 2.} $n \equiv 1 \pmod 3 \implies n^2 \equiv 1 \pmod{3}$ by Proposition ref{modprop}(3).

		\underline{Case 3.} $n \equiv 2 \pmod 3 \implies n^2 \equiv 4 \equiv 1 \pmod{3}$ by Proposition \ref{modprop}(3).

		Hence, proved.
	\end{proof}

	Since $a$ and $b$ are integers, by Lemma \ref{sqmod3}, $a^2$ and $b^2$ are either $0$ or $1$ modulo $3$.

	Let us assume both $a^2$ and $b^2$ are $1$ modulo $3$. Thus, by Proposition \ref{modprop}(3), we get that $c^2 \equiv a^2 + b^2 \equiv 2 \pmod{3}$. 

	But by Lemma \ref{sqmod3}, we know that $c^2 \not\equiv 2 \pmod{3}$. This is a contradiction. Therefore, either $a^2$ or $b^2$ is congruent to $0$ modulo $3$ which by Definition \ref{mod} means that either $3 \mid a^2$ or $3 \mid b^2$.

	Without loss of generality, let us assume that $3 \mid a^2$.
	By Lemma \ref{modprime}(3), since $3 \mid a \cdot a$ and $3$ is prime, we get either $3 \mid a$ or $3 \mid a$ which impiles $3 \mid a$.

	Therefore, either $a$ or $b$ must be divisible by $3$. Hence, proved.
\end{solution}


\begin{problem} Skipped. \end{problem}

\begin{problem} Suppose that $a$ and $b$ are positive integers, and $\gcd(a, b) = d$. Prove that $a \mid b$ if and only if $d = a$. To do this, here are the two things you should prove:
	
	(i) If $a \mid b$, then $d = a$.
	\begin{solution}
		Since $a \mid a$ and $a \mid b$, $a$ is a common divisor of $a$ and $b$.

		For any other common divisor $d' > 0$, we must have $d' \mid a$. Thus, $a = d'k$ for some integer $k$. Since $a$ and $d'$ are positive, then $k$ must be positive as well. This implies that $d' = \frac{a}{k}$ where $k$ is a positive intger. Hence, $d' \leq a$.

		Thus, $a$ is greater than any other common divisor of $a$ and $b$ which by Definition \ref{gcd} implies that $d = a$.

		Hence, proved.
	\end{solution}

	(ii) If $d = a$, then $a \mid b$.
	\begin{solution}
		Since $a$ is the greatest common divisor of $a$ and $b$, then by Definition \ref{gcd}, $a \mid b$.

		Hence, proved.
	\end{solution}

\end{problem}

\begin{problem}
	Prove that $m \equiv n \pmod{15}$ if and only if $m \equiv n \pmod{3}$ and $m \equiv n \pmod{5}$. To do that, prove the following,

	\bigbreak
	(a) If $m \equiv n \pmod{15}$, then $m \equiv n \pmod{3}$ and $m \equiv n \pmod{5}$
	\begin{solution}
		By Defintion \ref{mod}, we have 
		\begin{align}
			15 \mid (m-n) &\implies m-n = 15k \\
			\implies m-n = 5 \cdot 3k &\implies 5 \mid (m-n) \\
			\implies m-n = 3 \cdot 5k &\implies 3 \mid (m-n)
		\end{align}
		By Definition \ref{mod}, $m \equiv n \pmod{3}$ and $m \equiv n \pmod{5}$.
	\end{solution}

	(b) If $m \equiv n \pmod{3}$ and $m \equiv n \pmod{5}$, then $m \equiv n \pmod{15}$.
	\begin{solution}
		By Definition \ref{mod},  we have
		\begin{align}
			3 \mid (m-n) &\implies m - n = 3k \\
			5 \mid (m-n) &\implies 5 \mid 3k
		\end{align}
		By Lemma \ref{modprime}, since $\gcd(3,5)=1$, we get $5 \mid k$. Therefore, $k = 5l$ for some integer $l$ and $m-n = 15l \implies 15 \mid (m-n)$.

		Thus, by Definition \ref{mod}, $m \equiv n \pmod{15}$.

		Hence, proved.
	\end{solution}
\end{problem}


\begin{problem}
	Suppose that $a$ and $b$ are positive integers and $d = \gcd(a, b)$.

	\bigbreak
	(a) Prove that $\gcd(\frac{a}{d}, \frac{b}{d}) = 1$.
	\begin{solution}
		Let $g = \gcd(\frac{a}{d}, \frac{b}{d})$. Now, since $g$ is a divisor, we get
		$$\frac{a}{d} = gk \implies a = dgk$$
		$$\frac{b}{d} = gl \implies b = dgl$$
		for some integers $k$ and $l$.

		Now, by Definition \ref{div}, we know that $dg \mid a$ and $dg \mid b$. Therefore, $dg$ is a common divisor of $a$ and $b$.

		Since $d$ is the \emph{greatest common divisor} of $a$ and $b$, we can say that 
		$$dg \leq d \implies g \leq 1$$
		Since $g$ is a \emph{greatest common divisor} for $\frac{a}{d}$ and $\frac{b}{d}$, it must be atleast $1$ which means $g \geq 1$.

		The only number that satisfies both these conditions is $g = 1$.

		Hence, proved.
	\end{solution}

	(b) Prove that $\gcd(an, bn) = dn$ for every positive integer $n$.
	\begin{solution}
		Since $d$ is the \emph{greatest common divisor} of $a$ and $b$, we know that $a = dk$ and $b = dl$ for some integers $k$ and $l$.
		
		This implies that $an = dnk$ and $bn = dnl$. Therefore, by Definition \ref{div} and \ref{gcd}, $dn$ is a common divisor of $an$ and $bn$.

		Let us say $g = \gcd(an,bn)$. Now, by Theorem \ref{gcdalgo}, we know that $g = anx + any$ for some integers $x$ and $y$.
		This implies that $$g = n \cdot (ax + ay) \implies n \mid g$$

		Thus, we can say that $g = nq$ for some integer $q$. Now, by Definition \ref{div} anb \ref{gcd}, we have
		$$an = nqk', bn = nql' \implies a = qk', b = ql'$$
		Thus, $q$ is a common divisor of $a$ and $b$. Since $d$ is the \emph{greatest common divisor} of $a$ and $b$, we have
		$$q \leq d \implies qn \leq dn \implies g \leq dn$$

		But also, $dn$ is a common divisor of $an$ and $bn$, therefore, $dn \leq \gcd(an, bn)$. \\
		The only way both these conditions are satisified is when $dn = \gcd(an, bn)$.

		Hence, proved.
	\end{solution}
\end{problem}


\begin{problem}
	Assume that $a, b$ and $c$ are integers for which $\gcd(a, b) = 1$ and $\gcd(a, c) = 1$. Prove that $\gcd(a, bc) = 1$.
\end{problem}

\begin{solution}
	Let $g = \gcd(a, bc)$. Now, by Definition \ref{div} and \ref{gcd}, we have $g \mid a$ and $g \mid bc$.

	By Bezout's Identity ( \ref{gcdalgo} ), we know that for some integers $k$, $l$, $m$ and $n$, we have 
	\begin{align}
		ak + bl = 1 &\implies bl = 1 - ak \\
		am + cn = 1 &\implies cn = 1 - am \\
		\implies bl \cdot cn &= (1-ak) \cdot (1-am) \\
		\implies bc \cdot ln &= 1 - ak - am +a^2 km \\
		\implies a \cdot(k + m - akm) + bc \cdot ln &= 1
\end{align}
	
Thus, we have $ax + bcy = 1$ for some integers $x$ and $y$.
Since $\gcd(a, bc)$ must be the smallest positive integer with this property, by Bezout's Identity ( \ref{gcdalgo} , See Proof ), we get $\gcd(a, bc) \leq 1$. 

But also, since $1$ is a common divisor of $a$ and $bc$, we have $\gcd(a, bc) \geq 0$.

The only value that satisfies this is $\gcd(a, bc) = 1$.

Hence, proved.
\end{solution}


\begin{problem}Skipped.\end{problem}

\begin{problem}
	If $\gcd(a, b) = 1$, then we say that $\frac{a}{b}$ is in \emph{reduced form}. Prove that if $n$ is an integer then $$\frac{21n+4}{14n+3}$$ is in reduced form.
\end{problem}

\begin{solution}
	Here, we only need to show that $\gcd(21n+4, 14n+3) = 1$.

	We can easily show that 
	$$3 \cdot (14n+3) + (-2) \cdot (21n+4) = 42n + 9 - 42n - 8 = 1$$
	But since $\gcd(21n+4, 14n+3)$ is the smallest positive integer of the form $(21n+4)x + (14n+3)y$, we get that $\gcd(21n+4, 14n+3) \leq 1$.

	But also, $1$ is a common divisor of both so $\gcd(21n+4, 14n+3) \geq = 1$.
	
	This implies that $\gcd(21n+4, 14n+3) = 1$.

	Hence, proved.
\end{solution}


\begin{problem} 
	Prove that $3 \mid (4^n - 1)$ for any $n \in \mb{N}$ in two different ways.

	(a) First, prove it using modular arithmetic.
	\begin{solution}
		Since $4 \equiv 1 \pmod{3}$ and $n$ is an integer greater than or equal to $1$, we can say that $4^n \equiv 1^n \pmod{3}$ by repeated application of Proposition \ref{modprop}(3) as exponentiation is just repeated multiplication.

		This gives us $4^n \equiv 1 \pmod{3}$ for any $n \in \mb{N}$.
		By Definition \ref{mod}, we get that $3 \mid (4^n-1)$.
		Hence, proved.
	\end{solution}
	
	(b) Second, prove it using the fact
	$$x^n - y^n = (x-y)(x^{n-1} + x^{n-2}y + x^{n-3}y^2+ ... + xy^{n-2} + y^{n-1})$$ 
	for any real numbers $x$ and $y$.
	\begin{solution}
		Since $4$ and $1$ are real numbers, we can show that 
		\begin{align}
			4^n - 1^n &= (4-1)(4^{n-1} + 4^{n-2} \cdot 1 + ... + 4 \cdot 1^{n-2} + 1^{n-1}) \\
			\implies 4^n - 1 &= 3 \cdot (4^{n-1} + 4^{n-2} \cdot 1 + ... + 4 \cdot 1^{n-2} + 1^{n-1})
		\end{align}
		Since $4$ and $1$ are integers, the expression in parenthesis is also an integer. Thus, by Definition \ref{div}, we have $3 \mid (4^n - 1)$. Hence, proved.
	\end{solution}
\end{problem}


\begin{problem}
	Prove that every odd integer is a difference of two squares.
\end{problem}

\begin{scratch}
	$$5 = 3^2 - 2^2$$
	$$7 = 4^2 - 3^2$$
	$$9 = 5^2 - 4^2$$
	Here, we see a pattern that $2k+1 = (k+1)^2 - k^2$.
\end{scratch}

\begin{solution}
	Let $2k+1$ be an odd integer. Here, we can show that
	$$(k+1)^2 - k^2 = k^2 + 2k + 1 - k^2 = 2k + 1$$
	Thus, $2k+1$ is difference of squares of $k+1$ and $k$.
	Hence, proved.
\end{solution}


\begin{problem}
	Prove that for every positive integer $n$, there exist a string of $n$ consecutive intgers none of which are prime.
\end{problem}

\begin{solution}
	If we take the following numbers:
	$$(n+1)! + 2, (n+1)! + 3, (n+1)! + 4, ..., (n+1)! + (n+1)$$
	Here, the first number must be divisible by $2$, the second number must be divisible by $3$ and so on.
	There are in total $n$ integers in this sequence, all of which have a divisor other than $1$ and itself.

	Hence, proved.
\end{solution}

\begin{problem}Skipped.\end{problem}

\begin{problem}
	Suppose $n$ is an integer. Prove that if $n^2 \mid n$, then $n$ is either $-1, 0$ or $1$.
\end{problem}

\begin{solution}
	Since $n^2 \mid n$, we get that $n = kn^2$ for some integer $k$. Thus, 
	$$n = kn^2 \implies kn^2 - n = 0 \implies n(kn-1) = 0$$
	Therefore, either $n = 0$ or $k = \frac{1}{n}$. Since $k$ is an integer, the only value of $n$ for which $\frac{1}{n}$ is an integer is $1$ and $-1$.

	Thus, $n$ is either $-1, 0$ or $1$.

	Hence, proved.
\end{solution}

\begin{problem}
	As Evelyn Lamb pointed out,

	\bigbreak
	\emph{Every prime larger than $3$ is precisely $1$ off from a multiple of $3!$.}
	\bigbreak

	The above statement is true whether the "!" symbol is an exclamation or a factorial. Prove this.
\end{problem}

\begin{solution}
	For every prime $p > 3$, we must have $p \equiv 1 \pmod{3}$ or $p \equiv 2 \pmod{3}$ since it cannot be divisible by $3$.

	\underline{Case 1.} $p \equiv = 1 \pmod{3} \implies 3 \mid (p-1)$ by Definition \ref{mod}.

	\underline{Case 2.} $p \equiv = 2 \pmod{3} \implies p+1 \equiv 3 \equiv 0 \implies 3 \mid (p+1)$ by Definition \ref{mod}.

	Hence, proved.
	\bigbreak

	For every $p > 3$, we must have $p \in {1, 2, 3, 4, 5} \pmod{6}$ since $p \pmod{6}$ cannot be zero.

	\underline{Case 1.} $p \equiv 1 \pmod{6} \implies 6 \mid (p-1) \implies 3 \mid (p-1)$ by Definition \ref{mod} and Proposition \ref{transdiv} since $3 \mid 6$.
	
	\underline{Case 2.} $p \equiv 2 \implies 6 \mid (p-2) \implies 2 \mid (p-2) \implies 2 \mid p$ by Definition \ref{mod} and Proposition \ref{transdiv} and Lemma \ref{modprop}(1), thus, $p$ is not prime. \\
	This case is not possible.

	\underline{Case 3.} Similar to Case 2, $p \equiv 3 \implies 6 \mid (p-3) \implies 3 \mid (p-3) \implies 3 \mid p$, thus, $p$ is not prime. \\
	This case is not possible.

	\underline{Case 4.} $p \equiv 4 \implies 6 \mid (p-4) \implies 4 \mid (p-4) \implies 4 \mid p$, thus, $p$ is not prime. \\
	This case is not possible.

	\underline{Case 5.} $p \equiv 5 \pmod{6} \implies 6 \mid (p-5) \implies 6 \mid (p+1) \implies 3 \mid (p+1)$ by Definition \ref{mod} and Lemma \ref{modprop}(1).
\end{solution}

\begin{problem}
	Prove that $n \geq 2$ is not prime if and only if $n = st$ for some integers $s$ and $t$ where $1 < s, t < n$.
\end{problem}

\begin{solution}
	\underline{Case 1.} If $n \geq 2$ is not prime, then $n = st$ for some integers $s$ and $t$ where $1 < s, t < n$.
	
	Since $n$ is not prime, there must be a positive integer $s$ such that $s \mid n$, $s \neq 1$ and $s \neq n$ by Definition \ref{prime}.

	Now, since $s$ is a positive integer and $s \neq 1$, we must have $s > 1$.

	Since $s \mid n$, by Definition \ref{div}, we have $n = st$ for some integer $t$. Since both $s$ and $n$ are positive, $t$ must be positive as well.

	Now, since $s \neq n$, therefore, $t \neq 1$ which implies $t > 1$. Also, since $t > 1$, we must have $s = \frac{n}{t} < n$. 
	Similarly, since $s > 1$, we must have $t = \frac{n}{s} < n$. 

	Thus, we have $n = st$ for some integers $s$ and $t$ such that $1 < s,t < n$.

	\underline{Case 2.} If for some $n \geq 2$, $n = st$ for some integers $s$ and $t$ where $1 < s, t < n$, then $n$ is not prime.

	By Definition \ref{div}, we have $s \mid t$, $s > 0$, $s \neq 1$ and $s \neq n$. Therefore, by Definition \ref{prime}, $n$ is not a prime number.


	Hence, proved.
\end{solution}

\begin{problem}Skippped.\end{problem}
