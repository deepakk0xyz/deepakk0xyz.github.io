\section*{Exercises}


\bp Skipped.\ep

\bp
Suppose $A$ and $B$ are two boxes. Describe the following in terms of boxes: $A \\ B$, $\mc{P}(A)$ and $|A|$.
\ep
\bs
$A \\ B$: This is a box with all the objects in $A$ that are not in $B$.

$\mc{P}(A)$: This is a box with many boxes such that each box in this box has objects from $A$.

$|A|$: Number of objects in $A$.
\es


\bp Skipped.\ep
\bp Skipped.\ep
\bp Skipped.\ep
\bp Skipped.\ep
\bp Skipped.\ep
\bp Skipped.\ep
\bp Skipped.\ep

\bp
The set $\{5a+3b: a, b \in \mb{Z}\}$ is equal to a familiar set. By examining which elements are possible, determine the familiar set.
\ep

\bs
For $a = b = 0$, we get $5a + 3b = 0$. \\
For $a = -1$ and $b = 2$, we get $5a + 3b = -5+6 = 1$.

Since we have $0$ and $1$, we can build any integers $k$ by setting $a = -1 \cdot k$ and $b = 2 \cdot k$ which gives us $-5k+6k = k$.

Thus, our set is the set of integers, $\mb{Z}$.

It is trivial that all elements of the given set are integers to the given set is a subset of integers.

Now, for any $x \in \mb{Z}$, we can take $a = -x$ and $b = 2x$ since $-x, 2x \in \mb{Z}$ and get $5a + 3b = -5x + 6x = x$ which is an element of the given set.
Thus, $\mb{Z}$ is a subset of the given set.

Therefore, both sets are equal.

\es

\bp
Suppose $A, B$ and $C$ are sets. Is there a difference between $(A \times B) \times C$ and $A \times (B \times C)$? Explain your answer.
\ep

\bs
Let $(a, b, c) \in (A \times B) \times C$ such that $a \in A$, $b \in B$ and $c \in C$. All elements of $(A \times B) \times C$ are like this by Definition \ref{cartesian}.

Also, by defintion \ref{cartesian}, $(a, b, c) \in A \times (B \times C)$.

Therefore, $(A \times B) \times C \subseteq A \times (B \times C)$.
Similarly, we can show that $A \times (B \times C) \subseteq (A \times B) \times C$.

Hence, proved that these two sets are the same.
\es


\bp
Prove the second identity in De Morgan's Law ( Theorem \ref{demorgan} ).
That is, suppose $A$ and $B$ are subsets of $U$. Using $U$ as our universal set, show that
$$(A \cap B)^c  = A^c \cup B^c$$
\ep

\bs
Let $x \in (A \cap B)^c$ be any arbitrary element. By Definition \ref{complement}, we have that
\begin{align}
	x           & \in (A \cap B)^c        \\
	x           & \not\in A \cap B        \\
	x \not\in A & \text{ or } x \not\in B \\
	x \in A^c   & \text{ or } x \in B^c   \\
	x           & \in A^c \cup B^c
\end{align}
Hence, proved that $(A \cap B)^c \subseteq A^c \cup B^c$.

Now, let $x \in A^c \cup B^c$ be any arbitrary element.
\begin{align}
	x           & \in A^c \cup B^c        \\
	x \in A^c   & \text{ or } x \in B^c   \\
	x \not\in A & \text{ or } x \not\in B \\
	x           & \not\in A \cap B        \\
	x \in (A \cap B)^c
\end{align}

Hence, proved that $A^c \cup B^c \subseteq (A \cap B)^c$.

Therefore, these two sets must be equal.
\es


\bp Skipped.\ep
\bp Skipped.\ep

\bp  Suppse $A$ and $B$ are sets. Prove that
$$\mc{P}(A) \cup \mc{P}(B) \subseteq \mc{P}(A \cup B)$$
\ep

\bs
Let $X \in \mc{P}(A) \cup \mc{P}(B)$ be any arbitrary element. Then by Definition of union, we have
\begin{align}
	X & \in \mc{P}(A) \cup \mc{P}(B)              \\
	X & \in \mc{P}(A) \text{ or } X \in \mc{P}(B) \\
	X & \subseteq A \text{ or } X \subseteq B
\end{align}
By definition of union, we have $A \subseteq A \cup B$. Therefore, we can show that
$$X \subseteq A \cup B \implies X \in \mc{P}(A \cup B)$$

Thus, proved that $\mc{P}(A) \cup \mc{P}(B) \subseteq \mc{P}(A \cup B)$.
\es


\bp
Let $A = \{n \in \mb{Z} : 2 \mid n\}$, $B = \{ n \in \mb{Z} : 3 \mid n \}$ and $C = \{ n \in \mb{Z} : 6 \mid n \}$.

Show that $A \cap B = C$.
\ep

\bs
Let $n \in A \cap B$ be any arbitrary element. Then, we have
\begin{align}
	n        & \in A \cap B          \\
	n \in A  & \text{ and } n \in B  \\
	2 \mid n & \text{ and } 3 \mid n \\
	6        & \mid n                \\
	n \in C
\end{align}
Hence, proved that $A \cap B \subseteq C$.

Let $n \in C$ be any arbitrary element. Then, we have
\begin{align}
	n \in C \implies 6 \mid n & \implies n = 6k \text{ for some } k \in \mb{Z} \\
	n = 2(3k)                 & \text{ and } n = 3(2k)                         \\
	2 \mid n                  & \text{ and } 3 \mid n                          \\
	n \in A                   & \text{ and } n \in B                           \\
	n                         & \in A \cap B
\end{align}

Hence, proved that $C \subseteq A \cap B$.

Therefore, $A \cap B = C$.
\es

\bp
Let $A$ and $B$ are two sets. Prove that if $A \subseteq B$ then $\mc{P}(A) \subseteq \mc{P}(B)$.
\ep

\bs
Let us assume that $A$ and $B$ are two sets such that $A \subseteq B$.

Now, let $X \in \mc{P}(A)$ be any arbitrary element.
By Definiton \ref{powerset}, we have $X \subseteq A$.

Since $A \subseteq B$, we get that $X \subseteq B$. Thus, by Definition \ref{powerset}, we get that $X \in \mc{P}(B)$.

Hence, proved.
\es

\bp
Suppose $A, B$ and $C$ are sets with $C \neq \phi$.
\begin{enumerate}
	\item Prove that if $A \times C = B \times C$, then $A = B$.

	      \bs
	      Let $a \in A$ be any arbitrary element. Since $C \neq \phi$, there exists $c \in C$. Thus, by Definition \ref{cartesian}, $(a, c) \in A \times C$.

	      This implies that $(a, c) \in B \times C$ which means $a \in B$. Therefore, $A \subseteq B$.

	      Similarly, for any arbitrary element $b \in B$, we have $(b, c) \in B \times C \implies (b, c) \in A \times C \implies b \in A$. Therefore, $B \subseteq A$.

	      Hence, proved that $A = B$.
	      \es

	\item Explain why the condition $C \neq \phi$ is necessary.

	      \bs
	      If $C = \phi$ then catesian product is always $\phi$ and thus, any two sets can have equal cartesian products with $C$ and not be equal.

	      Also, if $C = \phi$, then there exists no element $c \in C$ and the argument above doesn't work.
	      \es

\end{enumerate}
\ep

\bp
Suppose $A, B$ and $C$ are sets. Prove that
$$A \cap (B \cup C) = (A \cap B) \cup (A \cap C)$$
\ep

\bs
Let $x \in A \cap (B \cup C)$ be any arbitrary element. Then, we have,
\begin{align}
	 & x \in A \cap (B \cup C)                                                   \\
	 & x \in A \text{ and } x \in B \cup C                                       \\
	 & x \in A \text{ and } ( x \in B \text{ or } x \in C )                      \\
	 & (x \in A \text{ and } x \in B) \text{ or } (x \in A \text{ and } x \in C) \\
	 & x \in A \cap B \text{ or } x \in A \cap C                                 \\
	 & x \in (A \cap B) \cup (A \cap C)
\end{align}

Hence, proved that $A \cap (B \cup C) \subseteq (A \cap B) \cup (A \cap C)$.

Since all the above steps are reversible, we can show that
$(A \cap B) \cup (A \cap C) \subseteq A \cap (B \cup C)$.

Hence, proved.
\es


\bp Skipped.\ep
\bp Skipped.\ep
\bp Skipped.\ep
\bp Skipped.\ep

\bp
Prove the following
\begin{enumerate}
	\item $\{5k+1 : k \in \mb{z} \} = \{ 5k + 6 : k \in \mb{Z} \}$

	      \bs
	      For any element $5k+1$ in the first set, we have
	      $$5k + 1 = 5k+6-5 = 5(k-1) + 6$$
	      Now, by definition $5(k-1)+6$ is in the second set.
	      Therfore, the first set is a subset of the second set.

	      For any element $5k+6$ in the second set, we have
	      $$5k+6 = 5k+5+1 = 5(k+1)+1$$
	      Now, by definition $5(k+1)+1$ is in the first set.
	      Therfore, the second set is a subset of the first set.

	      Hence, proved.
	      \es

	\item $\{ 12a + 3b : a, b \in \mb{Z} \} = \{ 3k : k \in \mb{Z} \}$

	      \bs
	      Since $\gcd(12, 3) = 3$, by Theorem \ref{gcdalgo}, there exists $m, n \in \mb{Z}$ such that $12m+3n = 3$.

	      Now, let $3k$ be any element in the second set. Since $3 = 12m + 3n$, we get $3k = (12m+3n)k = 12mk + 3nk$ which by definition is in the first set.
	      Thus, the second set is a subset of the first set.

	      Now, let $12a + 3b$ be any element in the first set. Then, we get $12a+3b = 3(4a+b)$ which by definition is in the second set.
	      Thus, the first set is a subset of the second set.

	      Hence, proved.
	      \es

	\item $\{8a+17b: a, b \in \mb{Z} \} = \mb{Z}$

	      \bs
	      Since $\gcd(8, 17) = 1$, by Theorem \ref{gcdalgo}, there exists $m, n \in \mb{Z}$ such that $8m+17n = 1$.

	      Now, let $k \in \mb{Z}$. Since $1 = 8m + 17n$, we get $k = (8m+17n)k = 8mk + 17nk$ which by definition is in the first set.
	      Thus, the second set is a subset of the first set.

	      Now, since every element in the first set is an integer, we know that, the first set is a subset of the second set.

	      Hence, proved.
	      \es
\end{enumerate}
\ep


\bp\label{conjectures}
Prove or find counterexample for the following:
\begin{enumerate}
	\item If $A \subseteq B \cup C$, then $A \cup B = B$ or $A \cup C = C$.

	      \bs
	      Let $A = \{1, 2, 3\}$ and $B = \{ 1, 3, 5 \}$ and $C = \{ 2, 4, 6 \}$.

	      Here, $B \cup C = \{1, 2, 3, 4, 5, 6\}$ and $A \subseteq B \cup C$.

	      But, $A \cup B = \{1, 2, 3, 5\} \neq B$ and $A \cup C = \{1, 2, 3, 4, 6\} \neq C$.
	      \es

	\item If $A \subseteq B \cup C$, then $A \cap B \subseteq B \cap C$.
	      \bs
	      Let us take the example from the previous problem.

	      Here, $A \cap B = \{1, 3\}$ and $B \cap C = \phi$. Thus, $A \cap B \not\subseteq B \cap C$.
	      \es

	\item If $A \subseteq B \cup C$, then $A \cap B \subseteq C$.
	      \bs
	      Let us take the example from the previous problem.

	      Here, $A \cap B = \{1, 3\} \not\subseteq C$.
	      \es


	\item If $A = B \setminus C$, then $B = A \cup C$.
	      \bs
	      Let $B = \{1, 2, 3\}$ and $C = \{2, 3, 4\}$. Therefore, $A = B \setminus C = \{1\}$.

	      Here, $A \cup C = \{1, 2, 3, 4\} \neq B$.
	      \es


	\item $A \setminus (B \cap C) = (A \setminus B) \cap (A \setminus C)$
	      \bs
	      Let $A = \{1, 2, 3, 4, 5\}$, $B = \{2\}$ and $C = \{2, 3\}$.

	      Here, $B \cap C = \{2\}$, $A \setminus B = \{1, 3, 4, 5\}$ and $A \setminus C = \{ 1, 4, 5 \}$.

	      Thus, $A \setminus (B \cap C) = \{1, 3, 4, 5\}$ and $(A \setminus B) \cap (A \setminus C) = \{1, 4, 5\}$ are not the same.
	      \es

	\item $(A \times B) \cup (C \times D) = (A \cup C) \times (B \cup D)$
	      \bs
	      Let $A = \{1\}, B = \{2\}, C = \{3\}$ and $D = \{4\}$.

	      Here, $A \times B = \{(1, 2)\}, C \times D = \{(3,4)\},
		      A \cup C = \{1, 3\}$ and $B \cup D \{2, 4\}$.

	      Thus, $(A \times B) \cup (C \times D) = \{(1, 2), (3, 4)\}$ and
	      $(A \cup C) \times (B \cup D) = \{(1, 2),(1, 4), (3, 2), (3, 4)\}$.

	      Hence, proved that these are not the same.
	      \es

	\item $(A \times B) \cap (C \times D) = (A \cap C) \times (B \cap D)$ \label{cart-inter}
	      \bs
	      Let $(x, y) \in (A \times B) \cap (C \times D)$ be an arbitrary element. Thus, we can show that

	      \begin{align}
		      (x, y) \in (A \times B) \cap (C \times D)                         \\
		      \implies (x, y) \in A \times B \text{ and } (x, y) \in C \times D \\
		      \implies x \in A, y \in B, x \in C \text{ and } y \in D           \\
		      \implies x \in A \cap C \text{ and } y \in B \cap D               \\
		      \implies (x, y) \in (A \cap C) \times (B \cap D)
	      \end{align}

	      This proves that the first set is the subset of the second.
	      Now, to prove the inverse, let $x \in (A \cap C) \times (B \cap D)$ be an arbitrary element. Then, we can show,

	      \begin{align}
		      (x, y) \in (A \cap C) \times (B \cap D)                           \\
		      \implies x \in A \cap C \text{ and } y \in B \cap D               \\
		      \implies x \in A, x \in C, y \in B \text{ and } y \in D           \\
		      \implies x \in A, y \in B \text{ and } x \in C, y \in D           \\
		      \implies (x, y) \in A \times B \text{ and } (x, y) \in C \times D \\
		      \implies (x, y) \in (A \times B) \cap (C \times D)
	      \end{align}

	      Hence, proved.
	      \es


	\item $\mc{P}(A) \cap \mc{P}(B) = \mc{P}(A \cap B)$
	      \bs
	      \begin{lemma}\label{intersub}
		      For any sets $X, A$ and $B$, if $X \subseteq A$ and $X \subseteq B$, then $X \subseteq A \cap B$.
	      \end{lemma}
	      \begin{proof}
		      Let $x \in X$ be any arbitrary element.
		      Since $X \subseteq A$, we know that $x \in A$.
		      Similarly, we know that $x \in B$.

		      Thus, by definition of intersection, $x \in A \cap B$.
		      Therefore, $X \subseteq A \cap B$. Hence, proved.
	      \end{proof}


	      Now, let $X \in \mc{P}(A) \cap \mc{P}(B)$ be an abitrary element. Then we have
	      \begin{align}
		      X \in \mc{P}(A) \cap \mc{P}(B)                                                             \\
		      X \in \mc{P}(A) \text{ and } X \in \mc{P}(B) &  & \quad \text{(Definition \ref{powerset})} \\
		      X \subseteq A \text{ and } X \subseteq B     &  & \quad \text{(Lemma \ref{intersub})}      \\
		      X \subseteq A \cap B                                                                       \\
		      X \in \mc{P}(A \cap B)
	      \end{align}

	      Thus, proved that $\mc{P}(A) \cap \mc{P}(B) \subseteq \mc{P}(A \cap B)$.

	      Now, let $X \in \mc{P}(A \cap B)$ be an abitrary element. Then by Definition \ref{powerset}, we have
	      $$X \in \mc{P}(A \cap B) \implies X \subseteq A \cap B$$

	      Since $A \cap B \subseteq A$ and $A \cap B \subseteq B$, we have,

	      \begin{align}
		      X \subseteq A \text{ and } X \subseteq B     \\
		      X \in \mc{P}(A) \text{ and } X \in \mc{P}(B) \\
		      X \in \mc{P}(A) \cap \mc{P}(B)
	      \end{align}

	      Thus, proved that $\mc{P}(A \cap B) \subseteq \mc{P}(A) \cap \mc{P}(B)$.

	      Hence, proved.
	      \es

	\item $\mc{P}(A) \setminus \mc{P}(B) = \mc{P}(A \setminus B)$
	      \bs
	      Let $A = B = \phi$. Now, $\mc{P}(A) = \mc{P}(B) = \{\phi\}$ and $A \setminus B = \phi$.

	      Therefore, $\mc{P}(A) \setminus \mc{P}(B) = \phi$ and $\mc{P}(A \setminus B) = \{\phi\}$ are not the same set.
	      \es

	\item $(A \times B) \times C = A \times (B \times C)$
	      \bs
	      Let us define $((a, b), c)$ to be the same as $(a, (b, c))$ as they are ordered pairs and write it as $(a, b, c)$.

	      Let $(a, b, c) \in (A \times B) \times C$ be any arbitrary element such that $a \in A, b \in B$ and $c \in C$.

	      By Definition \ref{cartesian}, we get $(b, c) \in (B \times C)$ which implies $(a, b, c) \in A \times (B \times C)$.

	      Thus, we have $(A \times B) \times C \subseteq A \times (B \times C)$.


	      Now, let $(a, b, c) \in A \times (B \times C)$ be any arbitrary element such that $a \in A, b \in B$ and $c \in C$.

	      By Definition \ref{cartesian}, we have $(a, b) \in A \times B$ which implies $(a, b, c) \in (A \times B) \times C$.

	      Thus, we have $A \times (B \times C) \subseteq (A \times B) \times C$.

	      Hence, proved.
	      \es
\end{enumerate}
\ep


\bp
Find counterexample of the conjecture $A \cup (B \cap C) = (A \cup B) \cap C$.

Note: This shows that parenthesis matter for this operation! Writing "$A \cup B \cap C$" is ambiguous.
\ep
\bs
Let $A = \{1\}, B = \{2,3\}$ and $C = \{2,4\}$.

Here, $B \cap C = \{2\}$ which implies $A \cup (B \cap C) = \{1,2\}$.

Also, $(A \cup B) = \{1,2,3\}$ which implies $(A \cup B) \cap C = \{2\}$.

Thus, the conjecture is false.
\es


\bp
Supppse somone conjectured that for any sets $A$ and $B$ with finitely many elements, we have $$|A \cup B| = |A| + |B|$$
If it is true, prove it. If it is false, find a counterexample and then conjecture a different formula for $|A \cup B|$.
\ep
\bs
Let $A = B = \{1\}$ then $A \cup B = \{1\}$.
Here, $|A \cup B| = 1$ and $|A| + |B| = 2$.
Thus, the conjecture is false.

The correct formula is $$|A \cup B| = |A| + |B| - |A \cap B|$$
\es

\bp
Let $A, B$ and $C$ be sets. Prove that if $A \cup B = A \cup C$ and $A \cap B = A \cap C$, then $B = C$.
\ep
\bs
Let $x \in B$ be any arbtrary element. Now, we have,
\begin{align*}
	x \in B \implies x \in A \cup B \\
	\implies x \in A \cup C         \\
	\implies x \in A \text{ or } x \in C
\end{align*}

Let us take two cases:

\underline{Case 1.} $x \in A$
\begin{align*}
	x \in A \implies x \in A \cap B &  & \quad \because x \in B             \\
	\implies x \in A \cap C         &  & \quad \because A \cap B = A \cap C \\
	\implies x \in C
\end{align*}

\underline{Case 2.} $x \in C$ is trivial.
\bigbreak

Thus, in both cases, we have shown that $x \in C$ and we can say that $B \subseteq C$.

The above steps are the same to show that $C \subseteq B$ since we can interchange $B$ and $C$ in the above.

Hence, proved.
\es

\bp Skipped. \ep
\bp Skipped. \ep
\bp Skipped. \ep
\bp Skipped. \ep

\bp Suppse $A$ and $B$ are sets. Prove that $A \subseteq B$ if and only if $A \setminus B = \phi$. \ep
\bs
Let us assume that $A \subseteq B$. Then let us take any arbitrary element $x \in A \setminus B$, which gives us
$$x \in A \setminus B \implies x \in A \text{ and } x \not\in B$$

Since $A \subseteq B$, we know that $x \in A \implies x \in B$. So we have $x \in B$ and $x \not\in B$.

No element can satisfy that condition, therefore, $A \setminus B = \phi$.

Now, let us assume that $A \setminus B = \phi$. Thus, we have that for all $x \in A$, $x \not\in B$ is false which implies for all $x \in A$, $x \in B$.

Thus, $A \subseteq B$.

Hence, proved.
\es

\bp Skipped. \ep

\bp Prove that $(\mb{N} \times \mb{Z}) \cap (\mb{Z} \times \mb{N}) = \mb{N} \times \mb{N}$. \ep
\bs
Since $\mb{N} \subseteq \mb{Z}$, we have $\mb{N} \cap \mb{Z} = \mb{N}$.

Using Problem \ref{conjectures} Conjecture \ref{cart-inter}, we can say that
$$(\mb{N} \times \mb{Z}) \cap (\mb{Z} \times \mb{N})
	= (\mb{N} \cap \mb{Z}) \cap (\mb{Z} \cap \mb{N})
	= \mb{N} \times \mb{N}$$

Hence, proved.
\es

\bp If $\mb{R} \times \mb{R}$ is our universal set, describe the elements in the set $( \mb{Q} \times \mb{Q})^c$. \ep

\bs This set contains all ordered pairs $(x, y) \in \mb{R} \times \mb{R}$ such that $(x, y) \not\in \mb{Q} \times \mb{Q}$, that is, either $x \not\in \mb{Q}$ or $y \not\in \mb{Q}$.

Thus, the set $(\mb{Q} \times \mb{Q})^c$ contains all order pairs $(x,y) \in \mb{R}\times\mb{R}$ such that either $x$ or $y$ is irrational.
\es

\bp Skipped. \ep
\bp Skipped. \ep

\bp Let $A$ and $B$ be sets. Prove that $\mc{P}(A \cap B) = \mc{P}(A) \cap \mc{P}(B)$. \ep
\bs
Let $X \in \mc{P}(A \cap B)$ be any arbitrary element. Then, by Definition \ref{powerset},
\begin{align*}
	X          & \in \mc{P}(A \cap B)                                                                                               \\
	\implies X & \subseteq A \cap B                                                                                                 \\
	\implies X & \subseteq A \text{ and } X \subseteq B
	           &                                            & \quad \because A \cap B \subseteq A \text{ and } A \cap B \subseteq B \\
	\implies X & \in \mc{P}(A) \text{ and } X \in \mc{P}(B)                                                                         \\
	\implies X & \in \mc{P}(A) \cap \mc{P}(B)
\end{align*}

Thus, we have $\mc P(A \cap B) \subseteq \mc P(A) \cap \mc P(B)$.

Now, let $X \in \mc P(A) \cap \mc P(B)$ be any arbitrary element. Then, by Defintion \ref{powerset}, we  have
\begin{align*}
	X          & \in \mc P(A) \cap \mc P(B)                                                        \\
	\implies X & \in \mc P(A) \text{ and } X \in \mc P(B)                                          \\
	\implies X & \subseteq A \text{ and } X \subseteq B                                            \\
	\implies X & \subseteq A \cap B                       &  & \quad \text{(Lemma \ref{intersub})} \\
	\implies X & \in \mc P(A \cap B)
\end{align*}

Thus, we have $\mc P(A) \cap \mc P(B) \subseteq \mc P(A \cap B)$.

Hence, proved.
\es

\bp
Let $C$ be any set
\begin{enumerate}
	\item Prove that there is a unique set $A \in \mc P(C)$ such that for every $B \in \mc P(C)$ we have $A \cup B = B$.

	      \bs
	      Let $B = \phi \in \mc P(C)$. Now, we have $A \cup \phi = A = B \phi$. Thus, we have $A = \phi$ as the only set that satisfies this propery.
	      \es

	\item Prove that there is a unique set $A \in \mc P(C)$ such that for every set $B \in \mc P(C)$ we have $A \cup B = A$

	      \bs
	      Let $B = C$. Now, we have $A \cup B = A \cup C$.
	      Since $A \in \mc P(C)$, we know that $A \subseteq C$, and therefore, $A \cup C = C$.

	      Thus, we get $A \cup B = A \cup C = C = A$. Thus, $A = C$ is the only set that satisfies this.
	      \es

	\item Prove that there is a unique set $A \in \mc P(C)$ such that for every set $B \in \mc P(C)$ we have $A \cap B = B$

	      \bs
	      Let $B = C$. Now, since $A \subseteq C$, we have $A \cap B = A \cap C = A = B = C$.

	      Thus, $A = C$ is the only set that satisfies this.
	      \es

	\item Prove that there is a unique set $A \in \mc P(C)$ such that for every set $B \in \mc P(C)$ we have $A \cap B = A$

	      \bs
	      Let $B = \phi$. Now, $A \cap B = A \cap \phi = \phi = A$.
	      Thus, $A = \phi$ is the only set that satisfies this.
	      \es
\end{enumerate}
\ep

\begin{definition}
	The \emph{symmetric difference} of sets $A$ and $B$ is the set
	$$A \Delta B = (A \cup B) \setminus (A \cap B)$$
\end{definition}
\begin{scratch}
	The definition above gives us:
	\begin{align*}
		x          & \in A \Delta B                               \\
		\implies x & \in (A \cup B) \setminus (A \cap B)          \\
		\implies x & \in A \cup B \text{ and } x \not\in A \cap B
	\end{align*}

	Thus, ($x \in A$ or $x \in B$) but $x \not\in A \cap B$.

	If $x \in A$ and $x \not\in A \cap B$, then we have $x \not\in B$.
	Similarly, if $x \in B$ and $x \not\in A \cap B$, then $x \not\in A$.

	Thus, we can say $x \in A, x  \not\in B$ or $x \in B, x \not\in A$.

	Hence, we have
	\begin{equation} \label{eq:2}
		A \Delta B = (A \setminus B) \cup (B \setminus A)
	\end{equation}

	Or similarly, $x \in A \text{ xor } x \in B$. Here "xor" means "exclusive or", that is, exactly one of these two is true.

	\begin{lemma}[Symmetric Difference]\label{lsd}
		If $x \in A \Delta B$, then either $x \in A$ or $x \in B$ but not both and not neither. In other words, $x \in A$ xor $x \in B$ where "xor" means "exclusive or".
	\end{lemma}

\end{scratch}

\bp Use the above definition to prove the following:
\begin{enumerate}
	\item Skipped.

	\item $(A \Delta B) \cup C = (A \cup C) \Delta (B \setminus C)$

	      \bs
	      \begin{named}[Part 1]
		      Let $x \in (A \Delta B) \cup C$. By Definition of Unions, we have three cases:

		      \underline{Case 1.} $x \in A \Delta B, x \not\in C$

		      Here, we have two subcases for $x \in A \Delta B$:

		      \underline{Case 1.1.} $x \in A, x \not\in B$

		      By Equation \ref{eq:2} and Definition \ref{complement}, we have,
		      \begin{align*}
			               & x \in A, x \not\in B, x \not\in C          \\
			      \implies & x \in A \cup C, x \not\in B \setminus C    \\
			      \implies & x \in (A \cup C) \setminus (B \setminus C) \\
			      \implies & x \in (A \cup C) \Delta (B \setminus C)
		      \end{align*}

		      \underline{Case 1.2.} $x \not\in A, x \in B$

		      By Equation \ref{eq:2} and Definition \ref{complement}, we have,
		      \begin{align*}
			               & x \not\in A, x \in B, x \not\in C          \\
			      \implies & x \not\in A \cup C, x \in B \setminus C    \\
			      \implies & x \in (B \setminus C) \setminus (A \cup C) \\
			      \implies & x \in (A \cup C) \Delta (B \setminus C)
		      \end{align*}


		      \underline{Case 2.} $x \not\in A \Delta B, x \in C$

		      By Lemma \ref{lsd}, we have two subcases for $x \not\in A \Delta B$:

		      \underline{Case 2.1} $x \not\in A, x \not\in B$

		      By Equation \ref{eq:2} and Definition \ref{complement}, we have,
		      \begin{align*}
			               & x \not\in A, x \not\in B, x \in C          \\
			      \implies & x \in A \cup C, x \not\in B \setminus C    \\
			      \implies & x \in (A \cup C) \setminus (B \setminus C) \\
			      \implies & x \in (A \cup C) \Delta (B \setminus C)
		      \end{align*}

		      \underline{Case 2.2} $x \in A, x \in B$

		      By Equation \ref{eq:2} and Definition \ref{complement}, we have,
		      \begin{align*}
			               & x \in A, x \in B, x \in C                  \\
			      \implies & x \in A \cup C, x \not\in B \setminus C    \\
			      \implies & x \in (A \cup C) \setminus (B \setminus C) \\
			      \implies & x \in (A \cup C) \Delta (B \setminus C)
		      \end{align*}


		      \underline{Case 3.} $x \in A \Delta B, x \in C$

		      By Lemma \ref{lsd}, we have two subcases for $x \in A \Delta B$:

		      \underline{Case 3.1.} $x \in A, x \not\in B$

		      By Equation \ref{eq:2} and Definition \ref{complement}, we have,
		      \begin{align*}
			               & x \in A, x \not\in B, x \in C              \\
			      \implies & x \in A \cup C, x \not\in B \setminus C    \\
			      \implies & x \in (A \cup C) \setminus (B \setminus C) \\
			      \implies & x \in (A \cup C) \Delta (B \setminus C)
		      \end{align*}

		      \underline{Case 3.2.} $x \not\in A, x \in B$

		      By Equation \ref{eq:2} and Definition \ref{complement}, we have,
		      \begin{align*}
			               & x \not\in A, x \in B, x \in C              \\
			      \implies & x \in A \cup C, x \not\in B \setminus C    \\
			      \implies & x \in (A \cup C) \setminus (B \setminus C) \\
			      \implies & x \in (A \cup C) \Delta (B \setminus C)
		      \end{align*}

		      Hence, proved that $(A \Delta B) \cup C \subseteq (A \cup C) \Delta (B \setminus C)$.
	      \end{named}

	      \begin{named}[Part 2]
					TODO


	      \end{named}
	      \es

	\item $(A \Delta B) \cap C = (A \cap C) \Delta (B \cap C)$


\end{enumerate}
\ep

\bp
Make a conjecture for the following sets are equal to. No need to prove your answer but explain your reasoning.
\begin{enumerate}

	\item
	      $$\bigcup_{n \in \mb{N}} \left[ 2 - \frac{1}{n}, 4 + \frac{1}{n} \right]$$
	      \bs
	      The number $2 - \frac{1}{n} \geq 1$ and $4 + \frac{1}{n} \leq 5$ for all $n \in \mb{N}$. Thus, the union is the closed interval $[1, 5]$.
	      \es


	\item
	      $$\bigcap_{n \in \mb{N}} \left[ 2 - \frac{1}{n}, 4 + \frac{1}{n} \right]$$
	      \bs
	      The number $2 - \frac{1}{n} < 2$ and $4 + \frac{1}{n} > 4$ for all $n \in \mb{N}$. Thus, the intersecion is the open interval $[2, 4]$.

	      For any number $x < 2$, we can find an $n$ such that $2 - \frac{1}{n} > x$ and thus, $x$ cannot be in the intersection. Similarly, for any number $x > 4$.
	      \es

	\item $$\bigcup_{n \in \mb{N}} \left[ 2 + \frac{1}{n}, 4 - \frac{1}{n} \right]$$
	      \bs
	      The number $2 + \frac{1}{n} > 2$ and $4 - \frac{1}{n} < 4$ for all $n \in \mb{N}$. Therefore, the union is the open interval $(2, 4)$.
	      \es


	\item $$\bigcap_{n \in \mb{N}} \left[ 2 + \frac{1}{n}, 4 - \frac{1}{n} \right]$$
	      \bs The intersection is $\{3\}$ since for $n = 1$, the interval becomse $[3,3]$. \es

\end{enumerate}
\ep
