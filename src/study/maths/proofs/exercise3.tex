\section*{Exercises}


\begin{problem}Skipped.\end{problem}

\begin{problem}
	Suppose $A$ and $B$ are two boxes. Describe the following in terms of boxes: $A \\ B$, $\mc{P}(A)$ and $|A|$.
\end{problem}
\begin{solution}
	$A \\ B$: This is a box with all the objects in $A$ that are not in $B$.

	$\mc{P}(A)$: This is a box with many boxes such that each box in this box has objects from $A$.

	$|A|$: Number of objects in $A$.
\end{solution}


\begin{problem}Skipped.\end{problem}
\begin{problem}Skipped.\end{problem}
\begin{problem}Skipped.\end{problem}
\begin{problem}Skipped.\end{problem}
\begin{problem}Skipped.\end{problem}
\begin{problem}Skipped.\end{problem}
\begin{problem}Skipped.\end{problem}

\begin{problem}
	The set $\{5a+3b: a, b \in \mb{Z}\}$ is equal to a familiar set. By examining which elements are possible, determine the familiar set.
\end{problem}

\begin{solution}
	For $a = b = 0$, we get $5a + 3b = 0$. \\
	For $a = -1$ and $b = 2$, we get $5a + 3b = -5+6 = 1$.

	Since we have $0$ and $1$, we can build any integers $k$ by setting $a = -1 \cdot k$ and $b = 2 \cdot k$ which gives us $-5k+6k = k$.

	Thus, our set is the set of integers, $\mb{Z}$.

	It is trivial that all elements of the given set are integers to the given set is a subset of integers.

	Now, for any $x \in \mb{Z}$, we can take $a = -x$ and $b = 2x$ since $-x, 2x \in \mb{Z}$ and get $5a + 3b = -5x + 6x = x$ which is an element of the given set.
	Thus, $\mb{Z}$ is a subset of the given set.

	Therefore, both sets are equal.

\end{solution}

\begin{problem}
	Suppose $A, B$ and $C$ are sets. Is there a difference between $(A \times B) \times C$ and $A \times (B \times C)$? Explain your answer.
\end{problem}

\begin{solution}
	Let $(a, b, c) \in (A \times B) \times C$ such that $a \in A$, $b \in B$ and $c \in C$. All elements of $(A \times B) \times C$ are like this by Definition \ref{cartesian}.

	Also, by defintion \ref{cartesian}, $(a, b, c) \in A \times (B \times C)$.

	Therefore, $(A \times B) \times C \subseteq A \times (B \times C)$.
	Similarly, we can show that $A \times (B \times C) \subseteq (A \times B) \times C$.

	Hence, proved that these two sets are the same.
\end{solution}


\begin{problem}
	Prove the second identity in De Morgan's Law ( Theorem \ref{demorgan} ).
	That is, suppose $A$ and $B$ are subsets of $U$. Using $U$ as our universal set, show that
	$$(A \cap B)^c  = A^c \cup B^c$$
\end{problem}

\begin{solution}
	Let $x \in (A \cap B)^c$ be any arbitrary element. By Definition \ref{complement}, we have that 
	\begin{align}
		x &\in (A \cap B)^c \\
		x &\not\in A \cap B \\
		x \not\in A &\text{ or } x \not\in B \\
		x \in A^c &\text{ or } x \in B^c \\
		x &\in A^c \cup B^c
	\end{align}
	Hence, proved that $(A \cap B)^c \subseteq A^c \cup B^c$.

	Now, let $x \in A^c \cup B^c$ be any arbitrary element.
	\begin{align}
		x &\in A^c \cup B^c \\
		x \in A^c &\text{ or } x \in B^c \\
		x \not\in A &\text{ or } x \not\in B \\
		x &\not\in A \cap B \\
		x \in (A \cap B)^c
	\end{align}

	Hence, proved that $A^c \cup B^c \subseteq (A \cap B)^c$.

	Therefore, these two sets must be equal.
\end{solution}


\begin{problem}Skipped.\end{problem}
\begin{problem}Skipped.\end{problem}

\begin{problem} Suppse $A$ and $B$ are sets. Prove that
	$$\mc{P}(A) \cup \mc{P}(B) \subseteq \mc{P}(A \cup B)$$
\end{problem}

\begin{solution}
	Let $X \in \mc{P}(A) \cup \mc{P}(B)$ be any arbitrary element. Then by Definition of union, we have
	\begin{align}
		X &\in \mc{P}(A) \cup \mc{P}(B) \\
		X &\in \mc{P}(A) \text{ or } X \in \mc{P}(B) \\
		X &\subseteq A \text{ or } X \subseteq B
	\end{align}
	By definition of union, we have $A \subseteq A \cup B$. Therefore, we can show that 
	$$X \subseteq A \cup B \implies X \in \mc{P}(A \cup B)$$

	Thus, proved that $\mc{P}(A) \cup \mc{P}(B) \subseteq \mc{P}(A \cup B)$.
\end{solution}


\begin{problem}
	Let $A = \{n \in \mb{Z} : 2 \mid n\}$, $B = \{ n \in \mb{Z} : 3 \mid n \}$ and $C = \{ n \in \mb{Z} : 6 \mid n \}$.

	Show that $A \cap B = C$.
\end{problem}

\begin{solution}
	Let $n \in A \cap B$ be any arbitrary element. Then, we have
	\begin{align}
		n &\in A \cap B \\
		n \in A &\text{ and } n \in B \\
		2 \mid n &\text{ and } 3 \mid n \\
		6 &\mid n \\
		n \in C
	\end{align}
	Hence, proved that $A \cap B \subseteq C$.

	Let $n \in C$ be any arbitrary element. Then, we have
	\begin{align}
		n \in C \implies 6 \mid n &\implies n = 6k \text{ for some } k \in \mb{Z} \\
		n = 2(3k) &\text{ and } n = 3(2k) \\
		2 \mid n &\text{ and } 3 \mid n \\
		n \in A &\text{ and } n \in B \\
		n &\in A \cap B
	\end{align}

	Hence, proved that $C \subseteq A \cap B$.

	Therefore, $A \cap B = C$.
\end{solution}

\begin{problem}
	Let $A$ and $B$ are two sets. Prove that if $A \subseteq B$ then $\mc{P}(A) \subseteq \mc{P}(B)$.
\end{problem}

\begin{solution}
	Let us assume that $A$ and $B$ are two sets such that $A \subseteq B$.

	Now, let $X \in \mc{P}(A)$ be any arbitrary element.
	By Definiton \ref{powerset}, we have $X \subseteq A$. 

	Since $A \subseteq B$, we get that $X \subseteq B$. Thus, by Definition \ref{powerset}, we get that $X \in \mc{P}(B)$.

	Hence, proved.
\end{solution}

\begin{problem}
	Suppose $A, B$ and $C$ are sets with $C \neq \phi$.
	\begin{enumerate}
	\item Prove that if $A \times C = B \times C$, then $A = B$.

	\begin{solution}
		Let $a \in A$ be any arbitrary element. Since $C \neq \phi$, there exists $c \in C$. Thus, by Definition \ref{cartesian}, $(a, c) \in A \times C$.

		This implies that $(a, c) \in B \times C$ which means $a \in B$. Therefore, $A \subseteq B$.

		Similarly, for any arbitrary element $b \in B$, we have $(b, c) \in B \times C \implies (b, c) \in A \times C \implies b \in A$. Therefore, $B \subseteq A$.

		Hence, proved that $A = B$.
	\end{solution}
	
	\item Explain why the condition $C \neq \phi$ is necessary.

	\begin{solution}
		If $C = \phi$ then catesian product is always $\phi$ and thus, any two sets can have equal cartesian products with $C$ and not be equal.

		Also, if $C = \phi$, then there exists no element $c \in C$ and the argument above doesn't work.
	\end{solution}

	\end{enumerate}
\end{problem}

\begin{problem}
	Suppose $A, B$ and $C$ are sets. Prove that 
	$$A \cap (B \cup C) = (A \cap B) \cup (A \cap C)$$
\end{problem}

\begin{solution}
	Let $x \in A \cap (B \cup C)$ be any arbitrary element. Then, we have,
	\begin{align}
		&x \in A \cap (B \cup C) \\
		&x \in A \text{ and } x \in B \cup C \\
		&x \in A \text{ and } ( x \in B \text{ or } x \in C ) \\
		&(x \in A \text{ and } x \in B) \text{ or } (x \in A \text{ and } x \in C) \\
		&x \in A \cap B \text{ or } x \in A \cap C \\
		&x \in (A \cap B) \cup (A \cap C)
	\end{align}

	Hence, proved that $A \cap (B \cup C) \subseteq (A \cap B) \cup (A \cap C)$.

	Since all the above steps are reversible, we can show that 
	$(A \cap B) \cup (A \cap C) \subseteq A \cap (B \cup C)$.

	Hence, proved.
\end{solution}


\begin{problem}Skipped.\end{problem}
\begin{problem}Skipped.\end{problem}
\begin{problem}Skipped.\end{problem}
\begin{problem}Skipped.\end{problem}

\begin{problem}
	Prove the following
	\begin{enumerate}
		\item $\{5k+1 : k \in \mb{z} \} = \{ 5k + 6 : k \in \mb{Z} \}$
			
		\begin{solution}
			For any element $5k+1$ in the first set, we have 
			$$5k + 1 = 5k+6-5 = 5(k-1) + 6$$
			Now, by definition $5(k-1)+6$ is in the second set.
			Therfore, the first set is a subset of the second set.

			For any element $5k+6$ in the second set, we have
			$$5k+6 = 5k+5+1 = 5(k+1)+1$$
			Now, by definition $5(k+1)+1$ is in the first set.
			Therfore, the second set is a subset of the first set.

			Hence, proved.
		\end{solution}
		
	\item $\{ 12a + 3b : a, b \in \mb{Z} \} = \{ 3k : k \in \mb{Z} \}$

		\begin{solution}
			Since $\gcd(12, 3) = 3$, by Theorem \ref{gcdalgo}, there exists $m, n \in \mb{Z}$ such that $12m+3n = 3$.

			Now, let $3k$ be any element in the second set. Since $3 = 12m + 3n$, we get $3k = (12m+3n)k = 12mk + 3nk$ which by definition is in the first set.
			Thus, the second set is a subset of the first set.

			Now, let $12a + 3b$ be any element in the first set. Then, we get $12a+3b = 3(4a+b)$ which by definition is in the second set.
			Thus, the first set is a subset of the second set.

			Hence, proved.
		\end{solution}

	\item $\{8a+17b: a, b \in \mb{Z} \} = \mb{Z}$

		\begin{solution}
			Since $\gcd(8, 17) = 1$, by Theorem \ref{gcdalgo}, there exists $m, n \in \mb{Z}$ such that $8m+17n = 1$.

			Now, let $k \in \mb{Z}$. Since $1 = 8m + 17n$, we get $k = (8m+17n)k = 8mk + 17nk$ which by definition is in the first set.
			Thus, the second set is a subset of the first set.

			Now, since every element in the first set is an integer, we know that, the first set is a subset of the second set.

			Hence, proved.
		\end{solution}
	\end{enumerate}
\end{problem}


\begin{problem}
	Prove or find counterexample for the following:
	\begin{enumerate}
		\item If $A \subseteq B \cup C$, then $A \cup B = B$ or $A \cup C = C$.
		
			\begin{solution}
				Let $A = \{1, 2, 3\}$ and $B = \{ 1, 3, 5 \}$ and $C = \{ 2, 4, 6 \}$.

				Here, $B \cup C = \{1, 2, 3, 4, 5, 6\}$ and $A \subseteq B \cup C$.

				But, $A \cup B = \{1, 2, 3, 5\} \neq B$ and $A \cup C = \{1, 2, 3, 4, 6\} \neq C$.
			\end{solution}
		
		\item If $A \subseteq B \cup C$, then $A \cap B \subseteq B \cap C$.
			\begin{solution}
				Let us take the example from the previous problem. 

				Here, $A \cap B = \{1, 3\}$ and $B \cap C = \phi$. Thus, $A \cap B \not\subseteq B \cap C$.
			\end{solution}

		\item If $A \subseteq B \cup C$, then $A \cap B \subseteq C$.
			\begin{solution}
				Let us take the example from the previous problem. 

				Here, $A \cap B = \{1, 3\} \not\subseteq C$.
			\end{solution}


		\item If $A = B \setminus C$, then $B = A \cup C$.
			\begin{solution}
				Let $B = \{1, 2, 3\}$ and $C = \{2, 3, 4\}$. Therefore, $A = B \setminus C = \{1\}$.

				Here, $A \cup C = \{1, 2, 3, 4\} \neq B$.
			\end{solution}


		\item $A \setminus (B \cap C) = (A \setminus B) \cap (A \setminus C)$
			\begin{solution}
				Let $A = \{1, 2, 3, 4, 5\}$, $B = \{2\}$ and $C = \{2, 3\}$.

				Here, $B \cap C = \{2\}$, $A \setminus B = \{1, 3, 4, 5\}$ and $A \setminus C = \{ 1, 4, 5 \}$.

				Thus, $A \setminus (B \cap C) = \{1, 3, 4, 5\}$ and $(A \setminus B) \cap (A \setminus C) = \{1, 4, 5\}$ are not the same.
			\end{solution}
			
		\item $(A \times B) \cup (C \times D) = (A \cup C) \times (B \cup D)$
			\begin{solution}
				Let $A = \{1\}, B = \{2\}, C = \{3\}$ and $D = \{4\}$.

				Here, $A \times B = \{(1, 2)\}, C \times D = \{(3,4)\}, 
				A \cup C = \{1, 3\}$ and $B \cup D \{2, 4\}$.

				Thus, $(A \times B) \cup (C \times D) = \{(1, 2), (3, 4)\}$ and 
				$(A \cup C) \times (B \cup D) = \{(1, 2),(1, 4), (3, 2), (3, 4)\}$.

				Hence, proved that these are not the same.
			\end{solution}

		\item $(A \times B) \cap (C \times D) = (A \cap C) \times (B \cap D)$
			\begin{solution}
				Let $(x, y) \in (A \times B) \cap (C \times D)$ be an arbitrary element. Thus, we can show that

				\begin{align}
					(x, y) \in (A \times B) \cap (C \times D) \\
					\implies (x, y) \in A \times B \text{ and } (x, y) \in C \times D \\
					\implies x \in A, y \in B, x \in C \text{ and } y \in D \\
					\implies x \in A \cap C \text{ and } y \in B \cap D \\
					\implies (x, y) \in (A \cap C) \times (B \cap D)
				\end{align}

				This proves that the first set is the subset of the second.
				Now, to prove the inverse, let $x \in (A \cap C) \times (B \cap D)$ be an arbitrary element. Then, we can show,

				\begin{align}
					(x, y) \in (A \cap C) \times (B \cap D) \\
					\implies x \in A \cap C \text{ and } y \in B \cap D \\
					\implies x \in A, x \in C, y \in B \text{ and } y \in D \\
					\implies x \in A, y \in B \text{ and } x \in C, y \in D \\
					\implies (x, y) \in A \times B \text{ and } (x, y) \in C \times D \\
					\implies (x, y) \in (A \times B) \cap (C \times D)
				\end{align}

				Hence, proved.
			\end{solution}


		\item $\mc{P}(A) \cap \mc{P}(B) = \mc{P}(A \cap B)$
			\begin{solution}
				\begin{lemma}\label{intersub}
					For any sets $X, A$ and $B$, if $X \subseteq A$ and $X \subseteq B$, then $X \subseteq A \cap B$.
				\end{lemma}
				\begin{proof}
					Let $x \in X$ be any arbitrary element.
					Since $X \subseteq A$, we know that $x \in A$.
					Similarly, we know that $x \in B$.

					Thus, by definition of intersection, $x \in A \cap B$. 
					Therefore, $X \subseteq A \cap B$. Hence, proved.
				\end{proof}


				Now, let $X \in \mc{P}(A) \cap \mc{P}(B)$ be an abitrary element. Then we have 
				\begin{align}
					X \in \mc{P}(A) \cap \mc{P}(B) \\
					X \in \mc{P}(A) \text{ and } X \in \mc{P}(B) && \quad \text{(Definition \ref{powerset})}\\
					X \subseteq A \text{ and } X \subseteq B && \quad \text{(Lemma \ref{intersub})} \\
					X \subseteq A \cap B \\
					X \in \mc{P}(A \cap B)
				\end{align}

				Thus, proved that $\mc{P}(A) \cap \mc{P}(B) \subseteq \mc{P}(A \cap B)$.

				Now, let $X \in \mc{P}(A \cap B)$ be an abitrary element. Then by Definition \ref{powerset}, we have
				$$X \in \mc{P}(A \cap B) \implies X \subseteq A \cap B$$

				Since $A \cap B \subseteq A$ and $A \cap B \subseteq B$, we have,

				\begin{align}
					X \subseteq A \text{ and } X \subseteq B \\
					X \in \mc{P}(A) \text{ and } X \in \mc{P}(B) \\
					X \in \mc{P}(A) \cap \mc{P}(B)
				\end{align}

				Thus, proved that $\mc{P}(A \cap B) \subseteq \mc{P}(A) \cap \mc{P}(B)$.

				Hence, proved.
			\end{solution}

		\item $\mc{P}(A) \setminus \mc{P}(B) = \mc{P}(A \setminus B)$
			\begin{solution}
				Let $A = B = \phi$. Now, $\mc{P}(A) = \mc{P}(B) = \{\phi\}$ and $A \setminus B = \phi$.

				Therefore, $\mc{P}(A) \setminus \mc{P}(B) = \phi$ and $\mc{P}(A \setminus B) = \{\phi\}$ are not the same set.
			\end{solution}

	\end{enumerate}
\end{problem}

