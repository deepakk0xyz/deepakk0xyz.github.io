\section*{Exercises}

\bp Prove that sum of the first $n$ odd natural numbers equals $n^2$ by induction or strong induction \ep

\bs 
We need to prove the following:
$$S_n : 1 + 3 + 5 + \ldots + (2n-1) = n^2$$

\underline{Base Case.} For $n = 1$, we have $1 = 1^2$ which is true.

\underline{Inductive Hypothesis.} Let $k \in \mb{N}$ and assume that 
$$1 + 3 + 5 + \ldots + (2k - 1) = k^2$$

\underline{Induction Step.} For $k+1$, we can show that 
\begin{align*}
	1 + 3 + 5 + \ldots + (2k - 1) + (2(k+1) - 1)
		&= k^2 + (2(k+1)-1) \\
		&= k^2 + 2k + 1 \\
		&= (k+1)^2
\end{align*}

Thus, we have prove that $S_k \implies S_{k+1}$.

\underline{Conclusion.} By induction, $S_n$ holds for every $n \in \mb{N}$, that is, sum of first $n$ odd natural numbers is $n^2$ for every $n \in \mb{N}$.
\es


\bp Prove thre proofs that if $n \in \mb{N}$, then $n^2 - n$ is even.
\begin{enumerate}
	\item Prove it by cases: considering $n$ is even or $n$ is odd.
		\bs
		Let $n \in \mb{N}$. Here, we have two cases:

		\underline{Case 1.} $n$ is even. 

		Since $n$ is even, then $n = 2k$ for some $k \in \mb{N}$. 
		Thus, we can show
		$$n^2 - n = (2k)^2 - 2k = 4k^2 - 2k = 2(2k^2-k)$$
		Thus, $n^2 - n$ is even if $n$ is even.

		\underline{Case 2.} $n$ is odd. 

		Since $n$ is even, then $n = 2k+1$ for some $k \in \mb{N}$. 
		Thus, we can show
		$$n^2 - n = (2k+1)^2 - (2k+1) = 4k^2 + 4k + 1 - 2k - 1 = 4k^2 + 2k = 2(2k^2 + k)$$
		Thus, $n^2 - n$ is even if $n$ is odd.

		Hence, proved.
		\es


	\item Prove it by applying Proposition \ref{triangular} to the sum $1 + 2 + 3 + \ldots + (n-1)$
		\bs
		By Proposition \ref{triangular}, we know that 
		$$1 + 2 + 3 + \ldots + (n-1) = \frac{(n-1)((n-1)+1)}{2} = \frac{n(n-1)}{2}$$
		$$\implies n^2 - n = 2 \cdot (1 + 2 + 3 + \ldots + (n-1))$$

		Thus, $n^n - n$ is even.
		\es
		
	\item Prove it by induction.
		\bs
		\underline{Base Case.} For $n = 1$, $1^2 - 1 = 0$ is even.

		\underline{Inductive Hypothesis.} Let $k \in \mb{N}$ and assume that $k^2 - k$ is even.

		\underline{Induction Step.} Here, we can show that 
		$$(k+1)^2 - (k+1) = k^2 + 2k + 1 - k - 1 = k^2 - k + 2k$$

		Since $k^2 - k$ is even and $2k$ is even, therefore, $k^2 - k + 2k$ is also even.
		Thus, $(k+1)^2 - (k+1)$ is even.

		\underline{Conclusion.} By Induction, $n^2 - n$ is even for all $n \in \mb{N}$.
		\es
\end{enumerate}
\ep


\bp Use induction or strong induction to prove the following for all $n \in \mb{N}$.
\begin{enumerate}
	\item $3 \mid (4^n - 1)$
		\bs
		\underline{Base Case.} For $n = 1$, $4^1 - 1 = 3$ is divisible by $3$.
		
		\underline{Inductive Hypothesis.} 
		For $k \in \mb{N}$, let $3 \mid (4^k-1)$.
		This implies that $4^k - 1 = 3m \implies 4^k = 3m+1$ for some $m \in \mb{Z}$.

		\underline{Induction Step.} 
		Now, for $k+1$, we can show
		$$4^{k+1}-1 = 4 \cdot 4^k - 1 = 4 \cdot (3m+1) - 1 = 12m + 3 = 3 \cdot (4m+1)$$
		Thus, $3 \mid 4^k-1$ implies $3 \mid 4^{k+1}-1$.
		
		\underline{Conclusion.} 
		By Induction, $3 \mid 4^n-1$ for all $n \in \mb{N}$.
		\es
	
	\item $6 \mid n^3 - n$
		\bs
		\begin{lemma}\label{lem} $k^2 + k$ is even.\end{lemma}
		\begin{proof}
			\underline{Base Case.} For $n = 1$, $1^2 + 1 = 2$ is even.
			
			\underline{Inductive Hypothesis.} Let $k \in \mb{N}$ and assume that $k^2 + k$ is even.

			\underline{Induction Step.} For $k+1$,
			$$(k+1)^2 - (k+1) = k^2 + 2k + 1 - k - 1 = (k^2 - k) + 2k$$
			Since $k^2 - k$ is even and $2k$ is also even, then $(k+1)^2 - (k+1)$ is even as well.

			\underline{Conclusion.} By Induction, $n^2 + n$ is even for all $n \in \mb{N}$.
		\end{proof}

		\underline{Base Case.} For $n = 1$, $1^3 - 1 = 0$ is divisible by $6$.

		\underline{Inductive Hypothesis.}
		For $k \in N$, let us assume that $6 \mid k^3 - k$. This implies that $k^3 - k = 6m$ for some $m \in \mb{Z}$.

		\underline{Induction Step.}
		For $k+1$, we can show
		$$(k+1)^3 - (k+1) = k^3 + 3k^2 + 3k + 1 - k - 1
		= (k^3 - k) + 3(k^2 + k)$$

		By Lemma \ref{lem}, $k^2 + k$ is even, that is, $k^2 + k = 2l$ for some $l \in \mb{Z}$. Thus, 

		$$(k+1)^3 - (k+1) = (k^3 - k) + 3(k^2 + k) = 6m + 3 \cdot 2l = 6(m+l)$$

		Hence, $6 \mid (k+1)^3 - (k+1)$.

		\underline{Conclusion.}
		By Induction, $6 \mid n^3 - n$ for all $n \in \mb{N}$.

		\es

	
	\item $9 \mid 3^{4n} + 9$
		\bs
		\underline{Base Case.} For $n = 1$, $3^{4 \cdot 1} + 9 == 81 + 9 = 90$ is divisible by $9$.

		\underline{Inductive Hypothesis.}
		For $k \in N$, let us assume that $9 \mid 3^{4k} + 9$. This implies that $3^{4k} + 9 = 9m \implies 3^{4k} = 9m-9$ for some $m \in \mb{Z}$.

		\underline{Inductive Step.}
		For $k+1$, we can show
		$$3^{4(k+1)} + 9 = 3^{4k+4} + 9 = 3^4 \cdot 3^{4k} + 9 
		81 \cdot 3^{4k} + 9 = 9 \cdot (9 \cdot 3^{4k} + 1)$$
		Hence, $9 \mid 3^{4(k+1)}+9$.

		\underline{Conclusion.}
		By Induction, $9 \mid 3^{4n} + 9$ for all $n \in \mb{N}$.

		\es

	\item $5 \mid n^5 - n$
		\bs
		\underline{Base Case.} For $n = 1$, $1^5 - 1 = 0$ is divisible by $5$.

		\underline{Inductive Hypothesis.}
		For $k \in N$, let us assume that $5 \mid k^5 - k$. This implies that $k^5 - k = 5m$ for some $m \in \mb{Z}$.

		\underline{Inductive Step.}
		For $k+1$, we can show
		\begin{align*}
			(k+1)^5 - (k+1) 
				&= k^5 + 5k^4 + 10k^3 + 10k^2 + 5k + 1 - k - 1 \\
				&= (k^5 - k) + 5 \cdot (k^4 + 2k^3 + 2k^2 + k) \\
				&= 5m + 5 \cdot (k^4 + 2k^3 + 2k^2 + k) \\
				&= 5 \cdot (m + k^4 + 2k^3 + 2k^2 + k) \\
		\end{align*}
		Hence, $5 \mid (k+1)^5 - (k+1)$.

		\underline{Conclusion.}
		By Induction, $5 \mid n^5 - n$ for all $n \in \mb{N}$.
		\es
		
	\item Skipped.
	\item Skipped.

\end{enumerate}
\ep


\bp Prove that each of the following hold for every $n \in \mb{N}$.
\begin{enumerate}[(a).]
	\item $\displaystyle 1^2 + 2^2 + 3^2 + \ldots + n^2 = \frac{n(n+1)(2n+1)}{6}$
		\bs
		\underline{Base Case.} 
		For $n = 1$, $1^1 = \frac{1 \cdot (1 + 1) \cdot ( 2 \cdot 1 + 1)}{6} = 1$ holds.
		
		\underline{Inductive Hypothesis.} Let $k \in \mb{N}$ and assume that 
		$$1^2 + 2^2 + 3^2 + \ldots + n^2 = \frac{n(n+1)(2n+1)}{6}$$

		\underline{Induction Step.} For $k+1$,
		\begin{align*}
			1^2 + 2^2 + 3^2 + \ldots + k^2 + (k+1)^2
				&= \frac{k(k+1)(2k+1)}{6} + (k+1)^2 \\
				&= \frac{k(k+1)(2k+1) + 6(k+1)^2}{6} \\
				&= \frac{(k+1)[k(2k+1) + 6(k+1)]}{6} \\
				&= \frac{(k+1)[2k^2 + 7k + 6]}{6} \\
				&= \frac{(k+1)[2k^2 + 3k + 4k + 6]}{6} \\
				&= \frac{(k+1)[k(2k+3) + 2(2k+3)]}{6} \\
				&= \frac{(k+1)(k+2)(2k+3)}{6} \\
				&= \frac{(k+1)((k+1)+1)(2(k+1)+1)}{6} \\
		\end{align*}

		Hence, proved for $k+1$.

		\underline{Conclusion.} By Induction, $1^2 + 2^2 + 3^2 + \ldots + n^2 = \frac{n(n+1)(2n+1)}{6}$ for all $n \in \mb{N}$.
		\es

	\item $\displaystyle 1^3 + 2^3 + 3^3 + \ldots + n^3 = \frac{n^2 (n+1)^2}{4}$
		\bs
		\underline{Base Case.} For $n = 1$, $1^3 =  \frac{1^2 (1+1)^2}{4} = 1$ is true.

		\underline{Inductive Hypothesis.} Let $k \in \mb{N}$ and assume that 
		$$1^3 + 2^3 + \ldots + k^3 = \frac{k^2 (k+1)^2}{4}$$

		\underline{Induction Step.} For $k+1$, we can show that
		\begin{align*}
			1^3 + 2^3 + \ldots + k^3 + (k+1)^3
				&= \frac{k^2 (k+1)^2}{4} + (k+1)^3 \\
				&= (k+1)^2 \left[ \frac{k^2}{4} + (k+1) \right] \\
				&= (k+1)^2 \left[ \frac{k^2 + 4(k+1)}{4} \right] \\
				&= (k+1)^2 \cdot \frac{(k+2)^2}{4} \\
				&= \frac{(k+1)^2 ((k+1)+1)^2}{4}
		\end{align*}

		Hence, proved.
		
		\underline{Conclusion.} By Induction, for all $n \in \mb{N}$,
		$$1^3 + 2^3 + \ldots + n^3 = \frac{n^2 (n+1)^2}{4}$$
		\es


	\item $\displaystyle 1 \cdot 2 + 2 \cdot 3 + 3 \cdot 4 + \ldots + n \cdot (n+1) = \frac{n(n+1)(n+2)}{3}$
		\bs
		\underline{Base Case.} For $n = 1$, $1 \cdot 2 = \frac{1 \cdot 2 \cdot 3}{3} = 2$ is true.

		\underline{Inductive Hypothesis.} Let $k \in \mb{N}$ and assume that 
		$$1 \cdot 2 + 2 \cdot 3 + 3 \cdot 4 + \ldots + k \cdot (k+1) = \frac{k(k+1)(k+2)}{3}$$

		\underline{Induction Step.} For $k+1$,
		\begin{align*}
			1 \cdot 2 + 2 \cdot 3 + 3 \cdot 4 + \ldots + &k \cdot (k+1) + (k+1) \cdot ((k+1)+1) \\
			&= \frac{k(k+1)(k+2)}{3} + (k+1)(k+2) \\
			&= (k+1)(k+2) \left( \frac{k}{3} + 1 \right) \\
			&= (k+1)(k+2) \left( \frac{k+3}{3} \right) \\
			&= \frac{(k+1)((k+1)+1)((k+1)+2)}{3}
		\end{align*}

		Hence, proved.

		\underline{Conclusion.} By Induction, for all $n \in \mb{N}$,
		$$1 \cdot 2 + 2 \cdot 3 + 3 \cdot 4 + \ldots + n \cdot (n+1) = \frac{n(n+1)(n+2)}{3}$$
		\es
		
	\item $\displaystyle 1 \cdot 3 + 2 \cdot 4 + 3 \cdot 5 + \ldots + n \cdot (n+2) = \frac{n(n+1)(2n+7)}{6}$
		\bs
		\underline{Base Case.} For $n = 1$, $1 \cdot 3 = \frac{1 \cdot (1+1) \cdot (2 \cdot 1 + 7)}{6} = \frac{1 \cdot 2 \cdot 9}{6} = 3$ is true.

		\underline{Inductive Hypothesis.} Let $k \in \mb{N}$ and assume that 
		$$1 \cdot 3 + 2 \cdot 4 + 3 \cdot 5 + \ldots + k \cdot (k+2) = \frac{k(k+1)(2k+7)}{6}$$

		\underline{Induction Step.} For $k+1$,
		\begin{align*}
			1 \cdot 3 + 2 \cdot 4 + 3 \cdot 5 + \ldots + &k \cdot (k+2) + (k+1) \cdot (k+3) \\
				&= \frac{k(k+1)(2k+7)}{6} + (k+1)(k+3) \\
				&= (k+1) \left[ \frac{k(2k+7)}{6} + (k+3) \right] \\
				&= (k+1) \left[ \frac{k(2k+7) + 6(k+3)}{6} \right] \\
				&= (k+1) \left[ \frac{2k^2 + 13k + 18}{6} \right] \\
				&= (k+1) \left[ \frac{(k+2)(2k+9)}{6} \right] \\
				&= \left[ \frac{(k+1)((k+1)+1)(2(k+1)+7)}{6} \right]
		\end{align*}

		Hence, proved.
		
		\underline{Conclusion.} By Induction, for all $n \in \mb{N}$,
		$$1 \cdot 3 + 2 \cdot 4 + 3 \cdot 5 + \ldots + n \cdot (n+2) = \frac{n(n+1)(2n+7)}{6}$$
		\es

	\item Skipped.

	\item $1 \cdot 1! + 2 \cdot 2! + 3 \cdot 3! + \ldots + n \cdot n! = (n+1)! - 1$
		\bs
		\underline{Base Case.} For $n = 1$, $1 \cdot 1! = 2! - 1 = 1$ is true.

		\underline{Inductive Hypothesis.} Let $k \in \mb{N}$ and assume that 
		$$1 \cdot 1! + 2 \cdot 2! + \ldots + k \cdot k! = (k+1)! - 1$$

		\underline{Induction Step.} For $k+1$,
		\begin{align*}
			1 \cdot 1! + 2 \cdot 2! + \ldots + & k \cdot k! + (k+1) \cdot (k+1)! \\
				&= (k+1)! - 1 + (k+1) \cdot (k+1)! \\
				&= (k+2) \cdot (k+1)! - 1 \\
				&= (k+2)! - 1
		\end{align*}

		Hence, proved.
		
		\underline{Conclusion.} By Induction, for all $n \in \mb{N}$,
		$$1 \cdot 1! + 2 \cdot 2! + \ldots + n \cdot n! = (n+1)! - 1$$
		\es
		
	\item Skipped.

	\item Skipped.
	
	\item Skipped.

	\item $\displaystyle \frac{1}{2!} + \frac{2}{3!} + \frac{3}{4!} + \ldots + \frac{n}{(n+1)!} = 1 - \frac{1}{(n+1)!}$
		\bs 
		\underline{Base Case.} For $n = 1$, $\frac{1}{2!} = 1 - \frac{1}{2!} = \frac{1}{2}$ is true.

		\underline{Inductive Hypothesis.} Let $k \in \mb{N}$ and assume that 
		$$\frac{1}{2!} + \frac{2}{3!} + \frac{3}{4!} + \ldots + \frac{k}{(k+1)!} = 1 - \frac{1}{(k+1)!}$$
		
		\underline{Indcution Step.} For $k+1$,
		\begin{align*}
			\frac{1}{2!} + \frac{2}{3!} + \ldots + &\frac{k}{(k+1)!} + \frac{k+1}{(k+2)!} \\
				&= 1 - \frac{1}{(k+1)!} + \frac{k+1}{(k+2)!} \\
				&= 1 - \frac{k+2}{(k+2)!} + \frac{k+1}{(k+2)!} \\
				&= 1 - \frac{1}{(k+2)!}
		\end{align*}

		Hence, proved.
		
		\underline{Conclusion.} By Induction, for any $n \in \mb{N}$,
		$$\frac{1}{2!} + \frac{2}{3!} + \ldots + \frac{n}{(n+1)!} = 1 - \frac{1}{(n+1)!}$$
		\es

	\item $\displaystyle \frac{1}{1 \cdot 2} + \frac{1}{2 \cdot 3} + \frac{1}{3 \cdot 4} + \ldots + \frac{1}{n \cdot (n+1)} = \frac{n}{n+1}$
		\bs 
		\underline{Base Case.} For $n = 1$, $\frac{1}{1 \cdot 2} = \frac{1}{1+1} = \frac{1}{2}$ is true.

		\underline{Inductive Hypothesis.} Let $k \in \mb{N}$ and assume that 
		$$\frac{1}{1 \cdot 2} + \frac{1}{2 \cdot 3} + \frac{1}{3 \cdot 4} + \ldots + \frac{1}{k \cdot (k+1)} = \frac{k}{(k+1)}$$
		
		\underline{Indcution Step.} For $k+1$,
		\begin{align*}
			\frac{1}{1 \cdot 2} + \frac{1}{2 \cdot 3} + \ldots + &\frac{1}{k \cdot (k+1)} + \frac{1}{(k+1) \cdot (k+2)} \\
				&= \frac{k}{(k+1)} + \frac{1}{(k+1)(k+2)} \\
				&= \frac{k(k+2)}{(k+1)(k+2)} + \frac{1}{(k+1)(k+2)} \\
				&= \frac{k^2 + 2k + 1}{(k+1)(k+2)} \\
				&= \frac{(k+1)^2}{(k+1)(k+2)} \\
				&= \frac{k+1}{k+2}
		\end{align*}

		Hence, proved.
		
		\underline{Conclusion.} By Induction, for any $n \in \mb{N}$,
		$$\frac{1}{1 \cdot 2} + \frac{1}{2 \cdot 3} + \ldots + \frac{1}{n \cdot (n+1)} = \frac{n}{(n+1)}$$
		\es
\end{enumerate}
\ep

\bp
Prove that each of the following holds for every $n \in \mb{N}$.
\begin{enumerate}[(a).]
	\item $n + 2 \leq 4n^2$
		\bs
		\underline{Base Case.} For $n = 1$, $1 + 1 = 2 < 4 \cdot 1^2 = 4$ is true.

		\underline{Induction Step.} Let $k \in \mb{N}$ and $k+2 < 4k^2$ then for $k+1$, 
		$$(k+1)+2 = (k+2)+1 < 4k^2 + 1 < 4k^2 + 8k + 4 = 4(k^2 + 2k + 1) = 4(k+1)^2$$
		Hence, proved.
		
		\underline{Conclusion.} By Induction, $n+2 < 4n^2$ for all $n \in \mb{N}$.
		\es

	\item $\displaystyle \frac{1}{\sqrt{1}} + \frac{1}{\sqrt{2}} + \ldots + \frac{1}{\sqrt{n}} \leq 2 \sqrt{n} - 1$
		\begin{scratch}
			In the inductive step, we will want to show that
			$$\frac{1}{\sqrt{1}} + \frac{1}{\sqrt{2}} + \ldots + \frac{1}{\sqrt{n}} + \frac{1}{\sqrt{n+1}} \leq 2 \sqrt{n+1} - 1$$

			We can show the above if we can show,
			$$2\sqrt{n} - 1 + \frac{1}{\sqrt{n+1}} &\leq 2 \sqrt{n+1} - 1$$

			Let's assume the above and see what we get,
			\begin{align*}
				2\sqrt{n} - 1 + \frac{1}{\sqrt{n+1}} &\leq 2 \sqrt{n+1} - 1 \\
				\implies 2\sqrt{n} + \frac{1}{\sqrt{n+1}} &\leq 2 \sqrt{n+1}\\
				\implies 2\sqrt{n(n+1)} + 1 &\leq 2(n+1) \\
				\implies 2\sqrt{n(n+1)} &\leq 2n + 1 \\
				\implies 4n^2 + 4n &\leq 4n^2 + 4n + 1 \\
				\implies 0 &\leq 1
			\end{align*}

			Since the above steps are reversible, we can prove the assumed statement.
		\end{scratch}
		\bs
		\underline{Base Case.} For $n = 1$, $\frac{1}{\sqrt{1}} = 1 \leq 2 \sqrt{1} - 1 = 1$ is true.
		\underline{Induction Step.} Let $k \in \mb{N}$ and assume that the statement is true for $k$. Then for $k+1$, 
		\begin{align*}
			\frac{1}{\sqrt{1}} + \frac{1}{\sqrt{2}} + \ldots + \frac{1}{\sqrt{k}} + \frac{1}{\sqrt{k+1}} 
				& \leq 2 \sqrt{k} - 1 + \frac{1}{\sqrt{k+1}} \\
				& = \frac{2 \sqrt{k(k+1)} + 1}{\sqrt{k+1}} - 1 \\
				& = \frac{\sqrt{4k^2 + 4k} + 1}{\sqrt{k+1}} - 1 \\
				& = \frac{\sqrt{(2k+1)^2 - 1} + 1}{\sqrt{k+1}} - 1 \\
				& \leq \frac{\sqrt{(2k+1)^2} + 1}{\sqrt{k+1}} - 1 \\
				& = \frac{2k+2}{\sqrt{k+1}} - 1 \\
				& = 2\sqrt{k+1} - 1
		\end{align*}

		Since $x-1 \leq x$ that implies $\sqrt{x-1} \leq \sqrt{x}$ for all $x \in \mb{N}$ and that is why the fourth step is valid.

		Hence, proved.
		
		\underline{Conclusion.} By Induction, for all $n \in \mb{N}$,
		$$\frac{1}{\sqrt{1}} + \frac{1}{\sqrt{2}} + \ldots + \frac{1}{\sqrt{n}} \leq 2\sqrt{n} - 1$$
		\es
		
	\item $\displaystyle 1 + \frac{n}{2} \leq \frac{1}{1} + \frac{1}{2} + \frac{1}{3} + \ldots + \frac{1}{2^n-1} + \frac{1}{2^n}$
		\bs
		\underline{Base Case.} For $n = 1$, the left hand side is $1 + \frac{1}{2} = \frac{3}{2}$ and the right hand side is $\frac{1}{1} + \frac{1}{2^1} = \frac{3}{2}$.
		Since $\frac{3}{2} \leq \frac{3}{2}$, the base case is true.

		\underline{Induction Step.} Let $k \in \mb{N}$ and assume that the statement is true for $k$. Now, for $k+1$, we have
		\begin{align*}
			1 + \frac{1}{2} + \frac{1}{3} + \ldots + \frac{1}{2^k-1} + \frac{1}{2^k} + \frac{1}{2^k+1} + \ldots + \frac{1}{2^{k+1} - 1} + \frac{1}{2^{k+1}} \\
			\leq 1 + \frac{k}{2} + \frac{1}{2^k+1} + \frac{1}{2^k+2} + \ldots + \frac{1}{2^{k+1}-1} + \frac{1}{2^{k+1}}
		\end{align*}

		Now, for the second part of the series, we can have show the following lemma:
		\begin{lemma}\label{one}
			For any $n \in \mb{N}$, 
			$$\frac{1}{2^n + 1} + \frac{1}{2^n + 2} + \ldots + \frac{1}{2^{n+1}-1} + \frac{1}{2^{n+1}} \geq \frac{1}{2}$$
		\end{lemma}
		\begin{proof}
			\underline{Base Case.} For $n = 1$, 
			$$\frac{1}{2^1+1} + \frac{1}{2^2} = \frac{1}{3} + \frac{1}{4} = \frac{7}{12} \geq \frac{1}{2}$$
			Thus, the base case is true.

			\underline{Induction Step.} Let $k \in N$ and assume that the above is true for $k$. Now, for $k+1$, we have the series
			$$\frac{1}{2^{k+1}+1} + \frac{1}{2^{k+1}+2} + \ldots + \frac{1}{2^{k+2}-1} + \frac{1}{2^{k+2}}$$
			Here, we can show that for every odd number between $2^{k+1}+1$ and $2^{k+2}$ in the form $2^{k+1}+2l-1$ for some $l \in \mb{N}$ and $l \leq 2^{k}$, that
			$$2^{k+1}+2l-1 \leq 2^{k+1}+2l \implies \frac{1}{2^{k+1}+2l-1} \geq \frac{1}{2^{k+1}+2l} = \frac{1}{2(2^k + l)}$$

			Thus, for the sequence above we can show that,
			\begin{align*}
				&\frac{1}{2^{k+1}+1} + \frac{1}{2^{k+1}+2} + \ldots + \frac{1}{2^{k+2}-1} + \frac{1}{2^{k+2}} \\
				&\geq \frac{1}{2^{k+1}+2} + \frac{1}{2^{k+1}+2} + \ldots + \frac{1}{2^{k+1}} + \frac{1}{2^{k+2}} \\
				&= \frac{2}{2^{k+1}+2} + \frac{2}{2^{k+1}+4} + \ldots + \frac{2}{2^{k+2}} \\
				&= \frac{1}{2^k + 1} + \frac{1}{2^k + 2} + \ldots + \frac{1}{2^{k+1}} \\
				&\geq \frac{1}{2}
			\end{align*}

			Hence, proved the statement for $k+1$.

			\underline{Conclusion.} By Induction, for all $n \in \mb{N}$,
			$$\frac{1}{2^n+1} + \frac{1}{2^n+2} + \ldots + \frac{1}{2^{n+1}-1} + \frac{1}{2^{n+1}} \geq \frac{1}{2}$$
		\end{proof}

		Now, by Lemma \ref{one},
		$$1 + \frac{k}{2} + \frac{1}{2^k+1} + \frac{1}{2^k+2} + \ldots + \frac{1}{2^{k+1}-1} + \frac{1}{2^{k+1}} \geq 1 + \frac{k}{2} + \frac{1}{2} = 1 + \frac{k+1}{2}$$

		Hence, proved the statement for $k+1$.
			

		\underline{Conclusion.} By Induction, for all $n \in \mb{N}$,
		$$1 + \frac{n}{2} \leq \frac{1}{1} + \frac{1}{2} + \ldots + \frac{1}{2^n-1} + \frac{1}{2^n}$$
		\es

	\item $2^n \leq 2^{n+1} - 2^{n-1} - 1$
		\bs
		\underline{Base Case.} For $n = 1$, $2^1 = 2 \leq 2^2 - 2^0 - 1 = 4-1-1 = 2$ holds.

		\underline{Induction Step.} Let $k \in \mb{N}$ and assume that the statement is true for $k$ which means
		$$2^k \leq 2^{k+1}-2^{k-1}-1 \implies 2^{k+1} \leq 2^{k+2}-2^k-2 \leq 2^{k+2}-2^k-1$$
		Hence, proved the statement for $k+1$.
		
		\underline{Conclusion.} By Induction, for all $n \in \mb{N}$, $2^n \leq 2^{n+1} - 2^{n-1} - 1$.
		\es

	\item $1 + 2^n \leq 3^n$
		\bs
		\underline{Base Case.} For $n = 1$, $1 + 2^1 = 3 \leq 3^1$ holds.

		\underline{Induction Step.} Let $k \in \mb{N}$ and assume that the statement is true for $k$ which means
		$$
		1 + 2^k \leq 3^k \implies 2 + 2^{k+1} \leq 3^k \cdot 2 
		\implies 1 + 2^{k+1} \leq 2 + 2^{k+1} \leq 3^k \cdot 2 \leq 3^{k+1}
		$$
		Hence, proved the statement for $k+1$.

		\underline{Conclusion.} By Induction, $1 + 2^n \leq 3^n$ for all $n \in \mb{N}$.
		\es

	\item $4^{n+4} \geq (n+4)^4$
		\bs
		\underline{Base Case.} For $n = 1$, $4^{1+4} = 4^5 = 1024 \geq (1+4)^4 = 625$ holds.

		\underline{Induction Step.} Let $k \in \mb{N}$ and assume that the statement is true for $k$ which means
		\begin{align*}
			(k+5)^4 &= ((k+4)+1)^4 = (k+4)^4 + 4(k+4)^3 + 6(k+4)^2 + 4(k+4) + 1 \\
							&= (k+4)^4 \left[ 1 + \frac{4}{k+4} + \frac{6}{(k+4)^2} + \frac{4}{(k+4)^3} + \frac{1}{(k+4)^4} \right]
		\end{align*}

		Here, we know that $k+4 \geq 5, (k+4)^2 \geq 25, (k+4)^3 \geq 125$ and $(k+4)^4 \geq 625$. Therefore, we can say that,
		\begin{align*}
			\frac{(k+5)^4}{(k+4)^4} &\leq \left( 1 + \frac{4}{5} + \frac{6}{25} + \frac{4}{125} + \frac{1}{625} \right) \\
			\implies \frac{(k+5)^4}{(k+4)^4} &\leq 4 = \frac{4^{k+5}}{4^{k+4}} \\
			\implies \frac{4^{k+5}}{4^{k+4}} &\geq \frac{(k+5)^4}{(k+4)^4} \\
		\end{align*}
		Now, since we have assumed that $4^{k+4} \geq (k+4)^4$, we can multiply both of these inequalities to get,
		$$4^{k+5} \geq (k+5)^4$$
		Hence, proved that statement is true for $k+1$.

		\underline{Conclusion.} By Induction, for all $n \in \mb{N}$, $4^{n+4} \geq (n+4)^4$.
	 	\es
\end{enumerate}
\ep

\bp
Make a conjecture for which $n \in \mb{N}$ have the property that $2^n < n!$. Then prove it by induction or strong induction.
\ep
\bs
For $n = 4$, we have $2^4 = 16 < 4! = 24$. Thus, it is true for all $n \geq 4$.

\underline{Base Case.} For $n = 4$, we have $2^4 = 16 < 4! = 24$.

\underline{Induction Step.} Lets assume that $k \in \mb{N}$ where $k \geq 4$ and $2^k < k!$. Now, for $k+1$, we know that $k+1 \geq 2$ so we can show that
$$(k+1)! = (k+1)\cdot k! \geq (k+1) \cdot 2^k \geq 2 \cdot 2^k = 2^{k+1}$$
Hence, proved the statement for $k+1$.

\underline{Conclusion.} By Induction, $2^k < k!$ for all $k \in \mb{N}$ and $k \geq 4$.
\es

\bp Skipped.\ep

\bp 
The \emph{Fermat Number} is defined as $2^{2^n} + 1$ for $n \geq 0$. Prove that for every $n \in \mb{N}_0$,
$$F_0 \cdot F_1 \cdot F_2 \cdot F_3 \cdot \ldots \cdot F_n = F_{n+1} - 2$$
\ep
\bs
\underline{Base Case.} For $n = 1$, $F_0 = 2^{2^0} + 1 = 2^1 + 1 = 3$ and $F_1 = 2^{2^1} + 1 = 2^2 + 1 = 5$. Thus, $F_0 = F_1 - 2$ holds.

\underline{Induction Step.} Let's assume $k \in \mb{N}_0$ and the statement is true for all $l \in \mb{N}_0$ where $l \leq k$. 

Now, for $k+1$, we have
\begin{align*}
	&F_0 \cdot F_1 \cdot F_2 \cdot \ldots \cdot F_k \cdot F_{k+1} 
		= (F_{k+1} - 2) \cdot F_{k+1} \\
	&= F_{k+1}^2 - 2 \cdot F_{k+1} 
		= (F_{k+1} - 1)^2 - 1 \\
	&= (2^{2^{k+1}})^2 - 1
		= 2^{2^{k+1} \cdot 2} - 1 = 2^{2^{k+2}} - 1 \\
	&= (F_{k+2} - 1) - 1
		= F_{k+2} - 2
\end{align*}

Hence, proved the statement for $k+1$.

\underline{Conclusion.} By Strong Induction, for all $n \in \mb{N}_0$, we have
$$F_0 \cdot F_1 \cdot F_2 \cdot \ldots \cdot F_n = F_{n+1} - 2$$
\es


\bp Skipped. \ep
\bp Skipped. \ep
