\section*{Exercises}

\bp Skipped. \ep
\bp Skipped. \ep

\bp Suppose that $n \in \mb{Z}$. Prove that if $n^2 - 4n + 7$ is even, then $n$ is odd. \ep
\bs
The contrapositive of the above is the following: "If $n$ is even, then $n^2 - 4n + 7$ is odd."

Since $n$ is even then $n = 2k$ for some integer $k$. Therefore, 

$$n^2 - 4n + 7 = 4k^2 - 8k + 7 = 2(2k^2 - 4k + 3) + 1$$

Here, we have shown that $n^2 - 4n + 7$ is odd.

Hence, proved by contrapositive.
\es


\bp Suppose that $m, n \in \mb{Z}$. Prove that if $mn$ is odd, then $m$ is odd and $n$ is odd. \ep
\bs
Contrapositive: If $m$ is even or $n$ is even, then $mn$ is even."

Without loss of generality, let's assume that $m$ is even.

Since $m$ is even, then $m = 2k$ for some integer $k$. This implies that $mn = 2kn = 2(kn)$. Hence, $mn$ is even.

Hence, proved by contrapositive.
\es

\bp Suppose $n \in \mb{Z}$. Prove the following:
\begin{enumerate}
	\item If $n^2$ is even, then $n$ is even.
		\bs
		Contrapositive: If $n$ is odd, then $n^2$ is odd.

		Since $n$ is odd, we have $n = 2k+1$ for some integer $k$.
		This implies that $n^2 = (2k+1)^2 = 4k^2 + 4k + 1 = 2(2k^2+2k)+1$.
		Therefore, $n^2$ is odd.

		Hence, proved by contrapositive.
		\es

	\item If $5n^2 + 3$ is even, then $n$ is odd.
		\bs
		Contrapositive: If $n$ is even, then $5n^2+3$ is odd.

		Since $n$ is even, we have $n = 2k$ for some integer $k$. 
		This implies that 
		$5n^2 + 3 = 5 \cdot 4k^2 + 3 = 2 \cdot 10k^2 + 3 = 2 \cdot ( 10k^2 + 1 ) + 1$.
		Therefore, $5n^2+3$ is odd.

		Hence, proved by contrapositive.
		\es

	\item If $n^2+4n+5$ is even, then $n$ is odd.
		\bs
		Contrapositive: If $n$ is even, then $n^2+4n+5$ is odd.

		Since $n$ is even, we have $n=2k$ for some integer $k$.
		This implies that $n^2 + 4n + 5 = 4k^2 + 8k + 5 = 2(2k^2 + 4k + 2) + 1$.
		Therefore, $n^2 + 4n + 5$ is odd.

		Hence, proved by contrapositive.
		\es
\end{enumerate}
\ep

\bp Suppose $n \in \mb{Z}$. Prove the following:
\begin{enumerate}
	\item If $4 \nmid n^2$, then $n$ is odd.
		\bs
		Contrapositive: If $n$ is even, then $4 \mid n^2$.

		Here, $n = 2k$ for some integer $k$ and $n^2 = 4k^2$ which implies $4 \mid n^2$. 

		Hence, proved by contrapositive.
		\es

	\item If $8 \nmid (n^2 - 1)$, then $n$ is even.
		\bs
		Contrapositve: If $n$ is odd, then $8 \mid (n^2 - 1)$.

		Let $n = 2k+1$ be an odd number. 
		Now, $n^2 - 1 = (2k+1)^2 - 1 = 4k^2 + 4k + 1 - 1 = 4k(k+1)$

		Now, we have two cases:

		\underline{Case 1.} $k$ is even.

		Here we have $k = 2m$ for some integer $m$.
		Then, $n^2 - 1 = 4 \cdot 2m \cdot (2m+1) = 8m \cdot (2m+1)$ which implies that $8 \mid (n^2 - 1)$.

		\underline{Case 2.} $k$ is odd.

		Here we have $k = 2m+1$ for some integer $m$.
		Then, $n^2 - 1 = 4 \cdot (2m+1) \cdot (2m+2) = 4 \cdot (2m+1) \cdot 2(m+1) = 8 \cdot (2m+1) \cdot (m+1)$ which implies that $8 \mid (n^2 - 1)$.

		Hence, proved by contrapositive.

		\es

	\item If $3 \nmid n^2$, then $3 \nmid n$.
		\bs
		Contrapositive: If $3 \mid n$, then $3 \mid n^2$.

		Let $n = 3k$ for some integer $k$. Then $n^2 = 9k^2 = 3 \cdot 3k^2$ which implies that $3 \mid n^2$.

		Hence, proved by conrapositive.
		\es

	\item If $3 \nmid (n^2-1)$, then $3 \mid n$.
		\bs
		Contrapositive: If $3 \nmid n$, then $3 \mid (n^2 - 1)$.

		Here, we have two cases:
		
		\underline{Case 1.} $n = 3k+1$ for some integer $k$.

		Thus, $n^2 - 1 = (3k+1)^2 - 1 = 9k^2 + 9k + 1 - 1 = 3(3k^2 + 3k)$ 
		implies that $3 \mid (n^2 - 1)$

		\underline{Case 2.} $n = 3k+2$ for some integer $k$.

		Thus, $n^2 - 1 = (3k+2)^2 - 1 = 9k^2 + 18k + 4 - 1 = 3(3k^2 + 6k + 1)$ 
		implies that $3 \mid (n^2 - 1)$

		Hence, proved by contrapositive.
		\es
\end{enumerate}
\ep

\bp Suppose $n, n, t \in \mb{Z}$. Prove the following:
\begin{enumerate}
	\item If $m^2(n^2+5)$ is even, then $m$ is even or $n$ is odd.
		\bs
		Contrapositive: If $m$ is odd and $n$ is even, then $m^2(n^2+5)$ is odd.

		Let $m = 2k+1$ and $n = 2l$ for some integers $k$ and $l$.
		Now, $m^2 = 4k^2 + 4k + 1$ is odd. And $n^2 + 5 = 4l^2 + 5$ is odd. 
		Since product of two odd numbers is odd, hence, $m^2(n^2+5)$ is odd.

		Hence, proved by contrapositive.
		\es

	\item If $(m^2 + 4)(n^2 - 2mn)$ is odd, then $m$ and $n$ are odd.
		\bs
		Contrapositive: If $m$ is even or $n$ is even, then $(m^2 + 4)(n^2 - 2mn)$ is even.

		\underline{Case 1.} $m$ is even.

		If $m$ is even, then so is $m^2 + 4$. Now, since product of an even number with any other number is even, then $(m^2 + 4)(n^2 - 2mn)$ is also even.

		\underline{Case 2.} $n$ is even.

		Here, $n^2 - 2mn = n(n-2m)$. Since product of an even number with another number is always even, therefore, $n(n-2m)$ is even.

		Similarly, $(m^2+4)(n^2-2mn)$ is also even.

		Hence, proved by contrapositive.
		\es

	\item If $m \nmid nt$, then $m \nmid n$ and $m \nmid t$.
		\bs
		Contrapositive: If $m \mid n$ or $m \mid t$, then $m \mid mt$.

		WLOG, let's assume that $m \mid n$, then there exists an integer $k$ such that $n = mk$. Now, $nt = mkt = m(kt)$ which implies that $m \mid nt$.

		Hence, proved by contrapositive.
		\es
\end{enumerate}
\ep


\bp Suppse $x, y \in \mb{R}$. Prove the following:
\begin{enumerate}
	\item If $x + y \geq 2$, then $x \geq 1$ or $y \geq 1$.
		\bs
		Contrapositive: If $x < 1$ and $y < 1$, then $x + y < 2$.

		This is trivial as the statement implies itself.

		Hence, proved by contrapositive.
		\es

	\item If $x^3 + xy^2 \leq y^3 + yx^2$, then $x \leq y$.
		\bs
		Contrapositive: If $x > y$, then $x^3 + xy^2 > y^3 + yx^2$.

		Since we have $x > y$ and we know that $x^2 + y^2 \geq 0$, we can say that

		$$x \cdot (x^2+y^2) > y \cdot (x^2+y^2) \implies x^3 + xy^2 > y^3 + yx^2$$

		Hence, proved by contrapositive.
		\es

	\item If $x^2 + 2x + \frac{1}{2} < 0$, then $x < 0$.
	\bs}
		Contrapositive: If $x \geq 0$, then $x^2 + 2x + \frac{1}{2} \geq 0$.

		Since $x \geq 0$, we get $x^2 \geq 0$ and $2x \geq 0$. Thus, 
		$$x^2 + 2x + \frac{1}{2} \geq \frac{1}{2} \geq 0$$
		
		Hence, proved by contrapositive.
		\es
		
	\item If $x^2 - 5x + 6 < 0$, then $2 < x < 3$.
		\bs 
		Contrapositive: If $x \leq 2$ or $x \geq 3$, then $x^2 - 5x + 6 \geq 0$.

		Let's note that $x^2 - 5x + 6 = (x-2)(x-3)$
		
		\underline{Case 1.} $x \leq 2$.

		Hence, $x-2 \leq 0$ and $x-3 \leq -1 < 0$. Since both are negative, we can say that $(x-2)(x-3) \geq 0$.

		\underline{Case 2.} $x \geq 3$

		Hence, $x-2 \geq 1 > 0$ and $x-3 \geq 0$. Since both are positive, we can say that $(x-2)(x-3) \geq 0$.

		Hence, proved by contrapositive.
		\es

	\item If $x^3 + x > 0$, then $x > 0$.
		\bs
		Contrapositive: If $x \leq 0$, then $x^3 + x \leq 0$.

		Here, since $x \leq 0$, we can say that $x^3 \leq 0$, which implies that $x^3 + x \leq 0$.

		Hence, proved by contrapositive.
		\es
\end{enumerate}
\ep


\bp
Consider three sets $A = \{1, 2, 3, 4\}$, $B = \{6, 7, 8, 9, 10\}$ and $C = \{1, 2, 3, 4, 5, 6, 7, 8, 9, 10\}$. Suppose that $n \in C$.
Prove that if $\frac{n(n+1)(2n+1)}{6}$ is divisible by $6$ then $n \in A \cup B$.
\ep
\bs
Contrapositive: If $n \not\in A \cup B$, then $\frac{n(n+1)(2n+1)}{6}$ is not divisible by $6$.

Since $n \in C$, then we have $n \in C \setminus (A \cup B)$.
We can show that $C \setminus (A \cup B) = \{ 5 \}$. 
Therefore, we have $n = 5$. For $n = 5$, we have
$$\frac{n(n+1)(2n+1)}{6} = \frac{5 \cdot 6 \cdot 11}{6} = 55$$
Now, $55$ is not divisible by $6$.

Hence, proved by contrapositive.
\es

