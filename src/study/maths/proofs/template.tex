\usepackage{amsthm}
\usepackage{amssymb}
\usepackage{mathtools}
\usepackage{thmtools}
\usepackage{tikz}

\declaretheorem[numberwithin=chapter]{theorem}
\declaretheorem[numberwithin=chapter]{principle}
\declaretheorem[numberwithin=chapter]{proposition}
\declaretheorem[numberwithin=chapter]{lemma}

\declaretheorem[numberwithin=chapter]{example}
\declaretheorem[numberwithin=chapter]{problem}
\declaretheorem[numberwithin=chapter,style=definition]{definition}

\declaretheorem[numbered=no,name=Scratch Work,style=definition]{scratch}
\declaretheorem[numbered=no,name=Proof Idea,style=definition]{proof idea}
\declaretheorem[numbered=no,style=definition]{solution}

\declaretheoremstyle[
  notefont=\bfseries,
  notebraces={}{},
  headformat=\NOTE]{named}

\declaretheorem[numbered=no,style=named]{named}

% \declaretheoremstyle[
%   headfont=\bgseries
% ]{proof}
% \declaretheorem[numbered=no,style=proof]{proof}

\newcommand{\mb}[1]{\mathbb{#1}}

\newcommand{\mc}[1]{\mathcal{#1}}

\newcommand{\bp}{\begin{problem}}
\newcommand{\ep}{\end{problem}}

\newcommand{\bs}{\begin{solution}}
\newcommand{\es}{\end{solution}}
